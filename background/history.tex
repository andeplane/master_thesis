\section{History of fluid mechanics and gas dynamics}
The history of any physical field is interesting in several ways. Physical questions usually start with one or more observations, the \textit{what}, and then the urge to understand \textit{why}. In science, \textit{why} is of course the question we ideally want to answer, but in many cases that is not achievable at first. An example is the statement of Kepler's three laws of planetary motion. Kepler had observations that he confirmed were indeed correct, but he did not know \textit{why} the planets behave like they do. Some 50 years later, Newton explained Kepler's three laws by his universal law og gravitation. This is the beauty of science, the \textit{why} is not required, just desired.\\
Another interesting part of the history is the amount of available information at the time of discoveries, which strongly affects their ability to develop new theories. Newton needed to create a theory that were in agreement with Kepler's laws, and Kepler needed his laws to agree with the observations that were done.\\
In this section, we will briefly discuss how the discoveries in fluid mechanics were done, and the questions that lead up to our current knowledge of the field. 

\subsection{The beginning in Greece}
The first scientific describtions of fluid mechanics dates back to Aristotle (384-322 B.C.) when he identified the continuum and dynamic drag in fluids.\cite{book:fluid_history} He wrote
\begin{quotation}
The continuous may be defined as that which is divisible into parts which are themselves divisible to infinity, as a body which is divisible in all ways. Magnitude divisible in one direction is a line, in three directions a body. And magnitudes which are divisible in this fashion are continuous. 
\end{quotation}
The idea of continuum is fundamental in most fluid mechanics theories and is a rather abstract concept that makes the mathematics work out beautifully. At the time of Aristotle, the mathematical framework was not yet established, so it was an impressive contribution to the field. The more intuitive drag force was described as
\begin{quotation}
It is impossible to say why a body that has been set in motion in a vacuum should ever come to rest. Why, indeed, should it come to rest at one place rather than another. As a consequence, it will either necessarily stay at rest, or if in motion, will move indefinitely unless some obstacle comes into collision with it.
\end{quotation}
For fluids in movement, the obstacle is the thing creating the drag force preventing the fluid to move freely. About a hundred years later, Archimedes (287-212 B.C.) published \textit{On Floating Bodies} where he discussed what is now known as \textit{Archimedes' principle} that states
\begin{quotation}
Any object, wholly or partially immersed in a fluid, is buoyed up by a force equal to the weight of the fluid displaced by the object.
\end{quotation}
Even today, more than 2000 years later, every high school student taking a physics course learn about Archimedes' principle. It is a simple and intuitive, yet remarkably powerful, statement that can easily be derived using Newtonian mechanics. 

\subsection{Conservation laws and the Navier-Stokes equations}
One can derive the Navier-Stokes equations (NSE) by assuming conservation of energy, mass and momentum ending up with\cite{batchelor2000introduction}
\begin{align}
	\rho \dpart {\vec v}{t} + \vec v\nabla\vec v = -\nabla p + \mu\nabla^2\vec v + \vec f,
\end{align}
where $\vec v$ is the fluid velocity, $\mu$ the viscosity and $\vec f$ is an external force (i.e. gravity). It is a set of coupled non-linear differential equations that can be seen as one vector differential equation. It has quite a few interesting analytically solvable solutions, but for most real systems, the geometry confining the fluid is so complex that it is solved on computers. A much used technique is to use a Finite Element Method (FEM) which works for arbitrary geometries. 

\subsection{The breakdown of contiinum}
A fundamental assumption in the NSE is that the space is continuous so that every point in space has well defined physical properties like density, velocity, temperature and pressure. This is known as the \textit{continuum hypothesis} and is invalid when the \textit{mean free path} $\lambda$, the average distance a particle moves between collisions, becomes large compared to some characteristic length $L$ in the system, i.e. the diameter of a channel. This property is quantified through the \textit{Knudsen number} which is defined as
\begin{align}
	Kn = \frac{\lambda}{L}.
\end{align}
For small Knudsen numbers (of order $10^{-2}$ or less), the contiinum hypthosis is valid and we can apply the Navier-Stokes equations\cite{karniadakis2005microflows}.
\subsection{Atomic models}
When the continuum hypothesis is invalid, we need another model describing the behaviour of the particles in our system. The first thing that might pop your mind might be to study the system at the atomic level. The physical set of rules that are controlling the atoms is of course quantum mechanics. The equations of motion and hence the dynamics of an atomic can in principle be calculated directly from quantum mechanics by solving Schrödinger's equation with perturbation theory, but the size of the system needs to be very small. An alternative, popular approach is to use a parameterized potential $U(\vec r^N)$, which is a function of the positions of all the atoms, and calculate the forces through the gradient of $U$. Newton's equations of motion is then integrated and the dynamics of the system are determined in a classical, deterministic way where important effects from quantum mechanics can be embedded in the potential. This method is called \textit{Molecular Dynamics} and is studied in chapter \ref{chap:md}. This method is computationally very expensive because it needs a detailed describtion of every atom in the system. For many problems, the specific details of each atom is not very important, but we want the statistical properties of the system.
\subsection{Statistical mechanics and the kinetic theory of gases}
We know from statistical mechanics that what dominates the macroscopic effects is the statistical behaviour of the system. One can derive the ideal gas law from pure statistical and combinatorial arguments with conservation of energy being the only physical concept\cite{ravndal2008statmech}. The idea is that it is the average properties of a system that are being observed wheras the actual state (i.e. all the positions and velocities of all atoms) of the system is not known. A physical system is often described as a distribution function $f(\vec r, \vec v; t)$ yielding the probability of finding an atom, or in general a particle in the volume element $d\vec r d\vec v$ at the time $t$. The dynamics of this function is described by the Boltzmann equation which is derived in section \ref{sec:boltzmann_equation}. From the Boltzmann equation, we can derive  the results we need to describe the properties of gases which will be input parameters to a model called Direct Simulation Monte Carlo (DSMC) which is studied in chapter \ref{chap:dsmc}. These results are called the kinetic theory of gases.
\subsection{Porous media}


\subsection{Nanoporous media}