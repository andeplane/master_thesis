\section{History}
The history of any physical field is interesting in several ways. Physical questions usually start with one or more observations, the \textit{what}, and then the urge to understand \textit{why} it happened. In science, \textit{why} is of course the question we ideally want to answer, but in many cases that is not achievable at first. An example is the statement of Kepler's three laws of planetary motion \ref{book:kepler}. Kepler had observations that he confirmed were indeed correct, but he did not know \textit{why} the planets behave like they do. Some 50 years later, Newton explained Kepler's three laws by his universal law og gravitation.\ref{TODO} This is the beauty of science, the \textit{why} is not required, just desired.\\
Another interesting part of the history is the amount of available information they had at the time, which strongly affected their ability to develop new theories. Newton needed to create a theory that were in agreement with Kepler's laws, and Kepler needed his laws to agree with the observations that were done. 

\subsection{Fluid mechanics}


\subsection{The breakdown of contiinum}
A fundamental assumption in the NSE is that the space is continuous so that every point in space has well defined physical properties such as density, velocity, temperature and pressure. This is known as the \textit{continuum hypothesis} and is invalid when the \textit{mean free path}, the average distance a particle moves between collisions, becomes large compared to some characteristic length in the system, i.e. the diameter of a channel. \ref{TODO} 
\subsection{Kinetic theory}
Instead of using conservation of mass, momentum and energy, one can use statistical mechanics and apply conservation of probability to derive the Boltzmann equation, which is an equation describing how a system evolve through time in the phase space. 
\subsection{Porous media}
\subsection{Nanoporous media}