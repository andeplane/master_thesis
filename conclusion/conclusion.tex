\section{Summary}
After having explained and discussed all models and the results we have obtained during our work, it is convenient to briefly summarize the thesis. The main physical problem we discussed in the beginning was shale gas extraction. In the fracturing process, the gas is released from within the shales through small pore networks, small channels with diameter from a few nanometers and up. Standard hydrodynamics breaks down at this scale because slip velocity and non-continuum effects, so we needed a model enabling us to study gas dynamics at this scale. In chapter \ref{chap:kinetic_theory} we briefly discussed statistical mechanics and kinetic theory, and used this to derive the Boltzmann equation, the mean free path and the mean collision time. These results were used to justify the DSMC model in chapter \ref{chap:dsmc}. We explained how the model is implemented in chapter \ref{chap:dsmc_implementation}. With the parallel implementation and the voxel-based representation of arbitrary geometries, we managed to achieve the first two goals in subsections \ref{goal:dsmc_1} and \ref{goal:dsmc_2}; a three-dimensional, parallel implementation of DSMC that works on arbitrary geometries. 

In chapter \ref{chap:dsmc_results}, we discussed the simulations with both model validation, performance tests and permeability analysis. The implementation shows a promising scaling performance up to at least 512 processors in addition to computing correct velocity profiles for a large range of Knudsen numbers compared to work of others. We confirmed that the Knudsen correction from section \ref{sec:knudsen_correction} predicted the permeability very well for cylinders, a geometry with a well defined Knudsen number. In the case of packed spheres, the Knudsen number is less obvious how to define due to statistical spread in the geometrical formation of the spheres, we found that an expectation value of the pore size could be used to find what we called an estimated Knudsen number (see appendix \ref{app:knudsen_number_packed_spheres}). We were able to predict the permeability to the correct order of magnitude, but the geometrical statistical spread was reflected through a spread in the final measured permeability. The ratio between the largest and the smallest permeabilities was a factor two or more.\\
We then moved on to discuss the second model we have studied, Molecular Dynamics, with an introduction to the model in chapter \ref{chap:md}. The numerical implementation was explained in detail in chapter \ref{chap:md_implementation}. We also introduced a new model for simulating fluids confined by a solid that interacts with realistic atomic forces and remains being a solid with the initial geometry by adding a harmonic oscillator potential on the atoms. The results in chapter \ref{chap:md_results} showed that the parallel implementation scaled satisfactory up to at least one thousand processors. We ran the MD program to measure the permeability in a cylinder for different Knudsen numbers, just like we did with DSMC. By simulating a similar system, but with a model based on different physical fundamentals, this is a good way to validate the Knudsen correction for both different length scales and a large range of Knudsen numbers. In MD, the cylinder was 30 times smaller than in DSMC (\unit{15}{\nano\meter} vs \unit{450}{\nano\meter}), with results confirming that the Knudsen correction works very well also for pores as small as 15 nanometers. Combining the results from both models, the goals in subsections \ref{goal:md_1} and \ref{goal:knudsen} were achieved. We will get back to the Knudsen correction in the discussion section below. We did not manage to implement the water-silica potential in MD as we wanted to do in subsection \ref{goal:md_2}. This would allow us to use MD to study flow in more realistic systems than with the Lennard-Jones potential. 
\\
In the last part of the thesis, the custom 3D visualization tool, we first gave an introduction to OpenGL and its purpose in chapter \ref{chap:opengl}. We discussed the basics of graphics programming as well as the more advanced, important parts of the rendering pipeline that allowed us to develop an extremely efficient visualization program to visualize large particle data sets with tens of millions of atoms rendered real-time with a decent frame rate. This was possible by using a combination of billboards and geometry shaders, both explained in chapter \ref{chap:particle_visualizer}. We concluded the part with a final show-off gallery in section \ref{sec:vis_gallery}. With this, we have obtained the last goal, described in subsection \ref{goal:vis}. 