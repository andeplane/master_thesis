\section{Summary}
After having explained and discussed all models and the results we have gathered during our work, it is convenient to briefly summarize the whole thesis. The main physical problem we discussed in the beginning was shale gas extraction. In the fracking process, the gas is released from everywhere in the shales through small pore networks, small channels with diameter from a few nanometers and up. Standard hydrodynamics breaks down at this scale because of non-continuum effects in addition to slip velocity, so we needed a model allowing us to study gas dynamics at this scale. In chapter \ref{chap:kinetic_theory} we briefly discussed statistical mechanics and kinetic theory, and used this to derive the Boltzmann equation, the mean free path and the mean collision time. These results were used to justify the assumptions the DSMC model in chapter \ref{chap:dsmc}. We explained how the model was implemented in chapter \ref{chap:dsmc_implementation}. With the parallel implementation and the voxel-based representation of arbitrary geometries, we managed to achieve the first two goals in subsections \ref{goal:dsmc_1} and \ref{goal:dsmc_2}.\\
In chapter \ref{chap:dsmc_results}, we discussed the simulations we performed with both model validation, performance tests and permeability analysis. We confirmed that the Knudsen correction from section \ref{sec:knudsen_correction} predicted the permeability very well for cylinders with a well defined Knudsen number. In the case of packed spheres, where the Knudsen number is less obvious how one would define due to statistical spread in the geometrical formation of the spheres, we found that an expectation value of the pore size could be used to find what we called an estimated Knudsen number (see appendix \ref{app:knudsen_number_packed_spheres}). We were able to predict the permeability to the correct order of magnitude, but the geometrical statistical spread was reflected through a spread in the final measured permeability. The ratio between the largest measured permeabilities and the smallest ones were a factor two or more, which greatly affects the economic outcome of gas extraction.\\
We then moved on to discuss MD with an introduction to the model in chapter \ref{chap:md}. The numerical implementation was explained in detail in chapter \ref{chap:md_implementation}. We also introduced a new model for simulating fluids confined by a solid that both interacted with realistic atomic forces and remained a solid with the initial geometry by adding a harmonic oscillator potential on the atoms. The rather short chapter with results (a lot of the discussions on concepts were already done in the DSMC results chapter), chapter \ref{chap:md_results}, showed that the parallel implementation scaled satisfactory up to at least one thousand processors. We also ran the MD program to measure the permeability in a cylinder for different Knudsen numbers, just like we did with DSMC. By simulating a similar system, but with a model based on different physical fundaments, this is a good way to validate the Knudsen correction for both different length scales and a large range of Knudsen numbers. In MD, the cylinder was 30 times smaller than in DSMC, with results confirming that the Knudsen correction works very well also for pores as small as 15 nanometers. With this, the goals in subsections \ref{goal:md_1} and \ref{goal:knudsen} were achieved. We will get back to the Knudsen correction in the discussion section below. We did not manage to implement the water-silica potential in MD as we wanted to do in subsection \ref{goal:md_2}. This would allow us to study flow in more realistic systems than with the Lennard-Jones potential we used.
\\
In the last part of the thesis, the custom 3d visualization tool, we first gave an introduction to OpenGL and its purpose in chapter \ref{chap:opengl}. We discussed the basics of graphics programming as well as the more advanced, important parts of the rendering pipeline that allowed us to develop an extremely efficient visualization program to visualize large particle data sets where tens of million atoms can be rendered realtime with a good framerate. This was possible by using billboards and geometry shaders, explained in chapter \ref{chap:particle_visualizer} with a final show-off gallery in the very last section \ref{sec:vis_gallery}. We have then obtained the goal in subsection \ref{goal:vis}.
\section{Discussion}
We have used two fundamentally different models to study flow in similar systems. MD is an atomic model computing forces between every atom whereas DSMC is a stochastic particle model based on statistical mechanics. Studying a physical problem this way has a great strength; when two models with different assumptions agree on the results, we have, to a greater extent, reasons to trust them than if it was just one model. But in any model, we make assumptions, and these may produce wrong answers. The main reason to use DSMC rather than MD is of course the reduced computational cost. The more physics you include in a model, the more computational expensive it is. This leads to the fundamental problem we always meet with before choosing a model - what is the physical problem, and what are the relevant phenomena?\\
A complete model of shale gas extraction should include physics describing both the gas production in the nanometer pores as well as the large scale flow from the fractures into the drilling hole. Today we do not have any good models covering the huge range in length scales and time scales. The models we have studied have indeed shown promising results predicting permeabilities in nanopores, but there might be other important effects we have not included in the models. For example, if the surfaces inside the pores have a non-zero net charge, it could significantly affect the fluid flow and the permeability. The model we used to create a solid in MD used a harmonic oscillator potential to keep the atoms at approximately the same position as they started. While this model includes realistic atomic forces and vibrations inside the solid, deformations and fracturing is impossible since the atoms are forced in the original formation.\\
It is assumed that the gas is trapped both inside already existing pores as well as adsorbed onto the organic material inside the shales. The rate of desorption from the organic material may require additional models releasing the gas from the organic material to the fracture network. It is needless to say that the models we have used are not complete in any sense. \\
From the theoretical perspective, we used the Knudsen correction to predict permeabilities in geometries with a closed form solution for the permeability at the no-slip scale - the absolute permeability. If we work with a geometry without a known absolute permeability, we can measure this in the high density limit and still use the Knudsen correction to predict permeabilities for dilute gases. The correction factor is a function of the Knudsen number, which is a well defined quantity in simple geometries like a cylinder. However, for more complicated geometries like the packed spheres, the Knudsen number should rather be seen as a distribution than one single number. It is possible, as we discuss in appendix \ref{app:knudsen_number_packed_spheres}, to calculate a distribution of Knudsen numbers providing both the mean value and the standard deviation which may be used to develop a \textit{stochastic} Knudsen correction based on the original model. 