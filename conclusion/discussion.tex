\section{Discussion}
We have used two fundamentally different models to study flow in similar systems. MD is an atomic model computing forces between every atom whereas DSMC is a stochastic particle model based on statistical mechanics. Studying a physical problem this way has a great strength; when two models with different assumptions agree on the results, we have, to a greater extent, reasons to trust them than if it was just one model. But in any model, we make assumptions, and these may produce wrong answers. The main reason to use DSMC rather than MD is of course the reduced computational cost. This points out an important fact; the more physics you include in a model, the more computational expensive it is. This leads to the fundamental problem we always meet before choosing a model; what is the physical problem, and what are the relevant phenomena?

A complete model of shale gas extraction should include physics describing both the gas production in the nanometer pores as well as the large scale flow from the fractures into the drilling hole. Today we do not have any good models covering this huge range in length scales and time scales. On the larger scale, we use continuum mechanics whereas the smaller scale requires models like MD or DSMC. On the smallest scale there are several alternatives to these two. Other popular methods are the lattice Boltzmann (LB), dissipative particle dynamics (DPD), which both can simulate larger systems for lager times. But again, a faster model usually includes less physics. A comparison between DSMC and these models would be interesting.

Both models we have studied have indeed shown promising results, calculating permeabilities in nanopores in agreement with the Knudsen correction, but there might be other important effects we have not included in the models. Such effects are probably not incorporated in the Knudsen correction either, but could become evident in an experiment. Validation of a model is not necessarily a validation of the assumptions. For example, if the surfaces inside the pores have a non-zero net charge, it could significantly affect the fluid flow and the permeability. We can incorporate this in MD by using a more advanced potential. The model we used to create a solid in MD used a harmonic oscillator potential to keep the atoms at approximately the same position as they started, keeping the geometrical form of the solid. While this model includes realistic atomic forces and vibrations inside the solid, deformations and fracturing is now impossible since the atoms are forced in the original formation.

It is also assumed that the gas is trapped both inside already existing pores as well as adsorbed onto the organic material inside the shales. The rate of desorption from the organic material may require additional models releasing the gas from the organic material to the fracture network. These are just a few, but important remarks about the limitations of the models. It is needless to say that they are not complete in any sense. Even if they \textit{were} complete, if they calculated both gas production and flow from the smallest pores into the fracture network, we would not be even close to a complete model of a shale reservoir. The coupling between different length scales - multi scale physics - is still an unsolved problem in physical science.

From the theoretical perspective, we used the Knudsen correction to predict permeabilities in geometries with a closed form solution for the absolute permeability at the no-slip scale. If we work with a geometry without a known absolute permeability, we can measure this in the high density limit, and still use the Knudsen correction to predict permeabilities for dilute gases. The correction factor is a function of the Knudsen number, which is a well defined quantity in simple geometries like a cylinder. However, for more complicated geometries like the packed spheres, the Knudsen number should rather be seen as a distribution than one single number. It is possible, as we discuss in appendix \ref{app:knudsen_number_packed_spheres}, to calculate a distribution of Knudsen numbers providing both the mean value and the standard deviation which may be used to develop a \textit{stochastic} Knudsen correction based on the original model. By obtaining statistical information about the geometry of a real shale reservoir, such a theory could in principle enable us to do an economical risk analysis based on the permeability distribution. 