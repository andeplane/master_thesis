\section{Da Boltzmann equation}
\label{sec:boltzmann_equation}
In tha section \ref{sec:kinetic_theory_distribution_function}, we introduced tha distribution function $f$ dat raps bout tha densitizzle up in a $6N$ dimensionizzle phase space. Now, if we know tha distribution function at $t=0$, we could up in principle compute any property of tha system. But at some later time tha distribution function might have chizzled, unless, of course, $\dm f/\dm t = 0$ up in which tha system would be up in a equilibrium state. We can, by applyin conservation of probability, derive a equation of motion fo' $f$. This equation is called tha Boltzmann equation n' raps bout how tha fuck $f$ evolves all up in time by assumin dat any chizzle of probabilitizzle round a point $(\vec r, \vec v)$ all up in tha time $t$ must be due to
\begin{itemize}
	\item flow all up in a surface up in tha phase space,
	\item a external force, or
	\item internal collisions.
\end{itemize}
All three will chizzle $f$ up in different ways. For simplicity, we will first assume dat tha particlez do not collide n' derive tha \textit{collisionless Boltzmann equation}. But do not worry, we will add tha collision term later n' end up wit tha full Boltzmann equation.

\subsection{Da collisionless Boltzmann equation}
Consider tha densitizzle $f(\vec r, \vec v, t)\dm\vec r\dm\vec v$ round tha phase space point $(\vec r,\vec v)$ all up in tha time $t$. Then, at a time $\dm t$ later, if we assume no forces n' dat tha total number of particlez has not chizzled, dat \textit{chunk} of densitizzle has moved ta $(\vec r + \vec v\dm t, \vec v)$. Conservation of probabilitizzle states dat any chizzle of $f$ within a volume $\Omega$ must flow all up in tha boundary $\partial \Omega$
\begin{align}
	\frac{\dm }{\dm t}\iint_\Omega\! f\, \dm \vec r \dm \vec v &= -\int_{\Omega_v}\!\dm \vec v\int_{\partial \Omega_r}\! f(\vec v\cdot \vec n_r)\, \dm S_r\\
	&= -\int_{\Omega_v}\!\dm \vec v\int_{\Omega_r}\! \nabla_\vec r\cdot(f\vec v)\, \dm \vec r = -\iint_{\Omega}\! \nabla_\vec r\cdot(f\vec v)\, \dm \vec r\dm \vec v
\end{align}
which becomes
\begin{align}
	\dpart{f}{t} + \nabla_\vec r \cdot(f\vec v) = \dpart{f}{t} + \vec v\cdot\nabla_\vec r f = 0
\end{align}
since $\vec v$ is independent of $\vec r$. We can extend dis equation by addin tha effectz of a external force $\vec F$ dat chizzlez tha velocitizzle up in tha same way as tha posizzle was chizzled above (except fo' tha factor $1/m$)
\begin{align}
	\label{eq:collisionless_boltzmann}
	\dpart{f}{t} + \vec v\cdot\nabla_\vec r f + \nabla_\vec v \cdot \frac{\vec F}{ m}f = 0,
\end{align}
which we call tha collisionless Boltzmann equation. I aint talkin' bout chicken n' gravy biatch. Well shiiiit, it aint nuthin but a phat approximation ta describe tha dynamics of straight-up dilute gases where intermolecular collisions occur rarely. But we should not ignore collisions between particles, so as promised, we will now peep dat by treatin collisions will step tha fuck up as a additionizzle term.
\subsection{Da collision operator}
\label{sec:boltzmann_collision_operator}
We consider a gangbangin' finger-lickin' dilute gas so we can assume dat only binary collisions occur (we ignore tha contribution from collisions between three or mo' particlez at a time). We also assume dat tha total juice, momentum n' mass is conserved up in all collisions. Then consider two particlez $i$ n' $j$ movin towardz each other wit velocitizzles $\vec v_i$ n' $\vec v_j$, n' relatizzle velocitizzle $\vec v_\text{rel} = \vec v_i - \vec v_j$. Us dudes define particle $i$ as tha \textit{incident} particle n' $j$ as tha \textit{target} particle fo' realz. After tha collision, tha particlez gonna git velocitizzles $\vec v_i'$ n' $\vec v_j'$ wit relatizzle velocitizzle $\vec v'_\text{rel} = \vec v_i' - \vec v_j'$.

In order ta make tha calculations simpler, we chizzle tha frame of reference, up in which our slick asses label tha velocitizzles wit a tilde so dat $\vec v \rightarrow \tilde{\vec v}$. If we chizzle tha target particle as initial frame of reference, we peep dat tha velocitizzle of tha incident particle becomes $\tilde {\vec v}_i = \vec v_\text{rel}$ n' $\tilde {\vec v}_i' = \vec v'_\text{rel}$. Right back up in yo muthafuckin ass. Since momentum is conserved, we know dat tha relatizzle velocitizzle must remain constant, $|\vec v_\text{rel}| = |\vec v'_\text{rel}|$, durin tha collision. I aint talkin' bout chicken n' gravy biatch. Da direction of $\vec v'_\text{rel}$ is given by tha anglez $\phi$ n' $\theta$ wit $\hat {\vec z}$ along $\vec v_\text{rel}$ n' $\phi \in [0, 2\pi], \theta \in [0, \pi]$. We can express $\vec v'_\text{rel}$ as
\begin{align}
	\vec v_\text{rel}' = \vec v_\text{rel} - 2\unitvector e(\unitvector e\cdot\vec v_\text{rel}),
\end{align}
where $\unitvector e$ be a arbitrary unit vector. Shiiit, dis aint no joke. If we multiply by $\unitvector e$, we peep dat 
\begin{align}
	\unitvector e\cdot \vec v_\text{rel}' = \unitvector e\cdot\left[\vec v_\text{rel} - 2\unitvector{e}(\unitvector e\cdot\vec v_\text{rel})\right] = -\unitvector e \cdot \vec v_\text{rel}
\end{align}
which gives tha symmetric relation
\begin{align}
	\vec v_\text{rel} = \vec v_\text{rel}' - 2\unitvector e(\unitvector e\cdot\vec v_\text{rel}').
\end{align}
Da angle between $\vec v_\text{rel}$ n' $\vec v_\text{rel}'$ is $\theta$, so
\begin{align}
	\vec v_\text{rel}'\cdot \vec v_\text{rel} = v_\text{rel}^2\cos\theta = v_\text{rel}^2(1 - 2\cos \chi),
\end{align}
where $\chi$ is tha angle between $\unitvector e$ n' $\vec v_\text{rel}$, which gives tha relation
\begin{align}
	\theta = \pi - 2\chi.
\end{align}
We now define tha solid angle element $\dm\Omega=\sin\theta \dm\theta \dm\phi$ bout $\vec v_\text{rel}'$ 
\begin{align}
	v_\text{rel} \dm\Omega &= v_\text{rel}\sin\theta \dm\theta \dm\phi = 2v_\text{rel}\sin(\pi - 2\chi)\dm\chi \dm\phi\\
	&= 4v_\text{rel}\cos\chi\sin\chi \dm\chi \dm\phi = 4\left|\unitvector e\cdot\vec v_\text{rel}\right|\sin\chi \dm\chi \dm\phi\\
	&= 4\left|\unitvector e\cdot\vec v_\text{rel}\right|\dm^2e,
\end{align}
where $\dm^2e = \sin\chi \dm\chi \dm\phi$ be a solid angle element bout $\unitvector e$. In tha following, we will calculate tha scatterin cross section which is tha \textit{area} dat raps bout tha likelihood of a incident particle bein scattered by tha target particle. Us dudes denote tha number densitizzle $\rho_n$ n' find dat tha incident flux is $\rho_n v_\text{rel}$. Da rate $h_{\dm\Omega}'$ of scattered particlez tha fuck into $\dm\Omega$ is
\begin{align}
	\label{eq:partial_scattering_rate}
	h_{\dm\Omega}' = \rho_n v_\text{rel}\sigma \dm\Omega,
\end{align}
where tha proportionalitizzle constant $\sigma$ is tha cross sectionizzle area. We might have nuff muthafuckin target particlez collidin independently of each other which will contribute ta tha scatterin rate. If our crazy asses have $n_t$ such particles, we obtain tha total scatterin rate $h_{\dm\Omega}$ by multiplyin \eqref{eq:partial_scattering_rate} by $n_t$
\begin{align}
	h_{\dm\Omega} = n_t\rho_n v_\text{rel}\sigma \dm\Omega.
\end{align}
Da \textit{differential} cross section $\sigma$ dependz on $\vec v_\text{rel}$ n' $\vec v_\text{rel}'$, so our phat asses denote it as $\sigma = \sigma(\vec v_\text{rel}\rightarrow \vec v_\text{rel}')$, whereas tha \textit{total} cross section $\sigma_T$ is given by integratin over all solid angles
\begin{align}
	\sigma_T = \int\! \sigma \, \dm\Omega.
\end{align}
Us thugs will now peep particlez wit velocitizzles up in tha range $[\vec v, \vec v + \dm\vec v]$ incident on target particlez wit velocitizzles up in tha range $[\vec v_1, \vec v_1 + \dm\vec v_1]$. Da incident flux is $gf(\vec v,\vec r)\dm\vec v$ n' tha number of target particlez is $f(v_1,\vec r)\dm\vec v_1\dm\vec r$. Da rate at which particlez wit velocitizzle $\vec v_1$ is scattered by particlez wit velocitizzle $\vec v$ is 
\begin{align}
	f(\vec v)f(\vec v_1) v_\text{rel} \sigma \, \dm\Omega \dm\vec v \dm\vec v_1 \dm\vec r = f(\vec v)f(\vec v_1)4\left|\unitvector e\cdot \vec v_\text{rel}\right| \sigma \,\dm^2e \dm \vec v \dm\vec v_1 \dm \vec r.
\end{align}
Da rate of loss is tha rate of which particlez up in $\dm\vec v\dm\vec r$ is bein hit by other particles. We can calculate dis by integratin over all solid anglez $\dm^2 e$ n' incident velocitizzles $\dm \vec v$
\begin{align}
	\text{rate of loss} = \left[\iiint\! f(\vec r, \vec v, t)f(\vec r, \vec v_1, t)4\left|\unitvector e \cdot \vec v_\text{rel} \right|\sigma \,\dm^2 e\dm \vec v_1\right]\dm \vec v\dm \vec r.
\end{align}
We also have tha inverse event, incident particlez wit velocitizzle $\vec v'$ hittin target particlez wit velocitizzle $\vec v_1'$ so dat tha final velocitizzle of tha target particlez is $\vec v$. This is calculated wit tha same idea
\begin{align}
	\text{rate of gain} = \left[\iiint\! f(\vec r, \vec v', t)f(\vec r, \vec v_1', t)4\left|\unitvector e \cdot \vec v_\text{rel} \right|\sigma \,\dm^2 e\dm \vec v_1\right]\dm\vec v\dm\vec r,
\end{align}
since $|\unitvector e\cdot \vec v_\text{rel}| = |\unitvector e\cdot \vec v_\text{rel}'|$ n' $\dm\vec v'\dm\vec v_1' = \dm\vec v\dm\vec v_1$. Da total chizzle up in tha distribution function is given by tha functionizzle $J[f]$
\begin{align}
	J[f] &= \iiint\! 4|\unitvector e\cdot\vec v_\text{rel}|\sigma[f'f_1' - ff_1]\,\dm\vec v_1 \dm^2e\\
	&= \iint\! v_\text{rel} \sigma[f'f_1' - ff_1] \,\dm\vec v_1 \dm\Omega,
\end{align}
where $f = f(\vec r, \vec v, t)$ n' $f_1 = f(\vec r, \vec v_1, t)$. Da full Boltzmann equation is then given by
\begin{align}
	\label{eq:boltzmann_equation}
	\dpart{f}{t} + \vec v\cdot \nabla_\vec r f + \frac{\vec F}{m}\cdot\nabla_\vec v f = J[f].
\end{align}
In tha derivation of tha collision operator $J[f]$, we assumed binary collisions only. This be a thugged-out decent approximation dat holdz fo' low densities. Put ya muthafuckin choppers up if ya feel dis! By definin tha force range $D$ n' tha dimensionless parameter $\nu = \rho_n D^3$, tha Boltzmann equation is valid when $\nu$ is lil' small-ass \cite{mclennan1989introduction}. $D^3$ defines a volume round a particle up in which tha forces cannot be neglected. Y'all KNOW dat shit, muthafucka! This type'a shiznit happens all tha time. This make $\nu$ tha average number of particlez within dat volume. If dat number is lil' small-ass (less than unity) then we can safely neglect collisions between three or mo' particles.
\section{\textit{H}-theorem}
Now our crazy asses have a integro-differential equation describin how tha fuck tha distribution function evolves all up in time. Us thugs will now peep Boltzmannz $H$-theorem which be a bangin result dat gives our asses all tha theoretical tools we need ta implement tha DSMC method. Y'all KNOW dat shit, muthafucka! Us thugs will use dis ta derive what tha fuck velocitizzle distribution a gas up in equilibrium obeys. In addition, we will find tha mean free path n' tha mean collision time which both is ghon be used up in tha Direct Simulation Monte Carlo method. Y'all KNOW dat shit, muthafucka! Let our asses define tha $H$-function as
\begin{align}
	\label{eq:h_function}
	H(t) = \langle \ln f \rangle = \iint\! f(\vec r, \vec v, t)\ln f(\vec r, \vec v, t)\,\dm \vec r \dm\vec v,
\end{align}
and differenciate wit respect ta time
\begin{align}
	\label{eq:h_theorem_one}
	\frac{\dm H}{\dm t} = \iint\! \dpart{f}{t}(\ln f) \,\dm\vec r \dm\vec v + \iint\! \dpart{f}{t} \,\dm\vec r \dm\vec v = \iint\! \dpart{f}{t}\ln f \,\dm\vec r \dm\vec v,
\end{align}
where tha last term vanished since tha number of particlez is conserved
\begin{align}
	\iint\! \dpart{f}{t} \,\dm\vec r \dm\vec v = \frac{\dm}{\dm t} \iint\! f \,\dm\vec r \dm\vec v = \frac{\dm N}{\dm t} = 0.
\end{align}
We multiply tha Boltzmann equation by $\ln f$ n' integrate over $\vec r$ n' $\vec v$
\begin{align}
	\nonumber
	\iint\! \ln f\dpart{f}{t} \,\dm \vec r \dm\vec v &= -\iint\! (\ln f)\vec v\cdot \nabla_\vec r f \,\dm\vec r \dm\vec v - \iint\! (\ln f) \frac{\vec F}{m}\cdot \nabla_\vec v f \,\dm\vec r\dm\vec v\\
	\label{eq:boltzmann_hell}
	&+ \iint\! \ln f J[f] \,\dm\vec r \dm\vec v.
\end{align}
Da first integral on tha right can be integrated by parts
\begin{align}
	\nonumber
	\iint\! (\ln f)\vec v\cdot \nabla_\vec r f \,\dm\vec r \dm\vec v &= \iint\! f(\ln f) (\vec v\cdot \hat{\vec n}) \,\dm \Gamma_\vec r\dm \vec v\\
	&- \iint\! f\ln f(\nabla_\vec r \cdot \vec v)\,\dm \vec r\dm \vec v = 0,
\end{align}
where $\dm \Gamma_\vec r$ indicates tha boundary of tha spatial domain. I aint talkin' bout chicken n' gravy biatch. Da integral is zero if we assume dat $f$ is zero all up in tha boundaries n' $\vec v$ is independent of $r$. Right back up in yo muthafuckin ass. Similarly wit tha second integral on tha right up in equation \eqref{eq:boltzmann_hell}
\begin{align}
	\nonumber
	\frac{1}{m}\iint\! (\ln f) \vec F\cdot \nabla_\vec v f \,\dm\vec r\dm\vec v &= \frac{1}{m}\iint\! f\ln f (\vec F\cdot \hat{\vec n}) \,\dm \Gamma_\vec v\dm \vec r\\
	&- \frac{1}{m}\iint\! f\ln f (\nabla_\vec v \cdot \vec F) \,\dm \vec r\dm \vec v = 0,
\end{align}
where $\dm \Gamma_\vec v$ indicates tha boundary of tha velocitizzle space. This integral be also zero since $f$ is zero when $\vec v\rightarrow \pm \infty$ n' tha force is independent of tha velocity. By recognizin dat tha left hand side of equation \eqref{eq:boltzmann_hell} straight-up is $\dm H/\dm t$, we end up with
\begin{align}
 	\frac{\dm H}{\dm t} = \iiint\! \ln f [f'f_1' - ff_1] g \sigma \,\dm\Omega\dm\vec v_1 \dm\vec v,
\end{align}
which can be freestyled as\cite{mcquarrie1973statistical}
\begin{align}
	\label{eq:h_theorem_integral}
 	\frac{\dm H}{\dm t} = \frac{1}{4}\iiint\! \ln\left[\frac{ff_1}{f'f_1'}\right] [f'f_1' - ff_1]g \sigma \, \dm\Omega\dm\vec v_1 \dm\vec v.
\end{align}
Da integrand iz of tha form $-(x-y)\ln(x/y)$ which is wack fo' $x\neq y$ n' zero fo' $x=y$. This means that
\begin{align}
	\frac{\dm H}{\dm t} \leq 0,
\end{align}
which is called tha $H$-theorem. Right back up in yo muthafuckin ass. Since $H(t)$ is bounded (see equation \eqref{eq:h_function} n' remember dat $f$ indeed is bounded), up in tha limit $t\rightarrow\infty$, $H(t)$ reaches a equilibrium state as $\dm H/\dm t = 0$. This up in turn means dat tha integrand up in \eqref{eq:h_theorem_integral} must be zero as well which happens if
\begin{align}
	\label{eq:equilibrium_f_relation}
	f'f_1' = ff_1,
\end{align}
which allows our asses calculate tha equilibrium velocitizzle distribution.
