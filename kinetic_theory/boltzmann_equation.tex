\section{The Boltzmann equation}
The Boltzmann equation is nothing more than a conservation law for the distribution function $f_N$. It describes how $f_N$ evolves through time by using that any change of probability must be due to one of the following
\begin{itemize}
	\item flow through a surface
	\item an external force
	\item internal collisions
\end{itemize}
All three will change $f_N$ in different ways. We will first assume no collisions between molecules and derive the collisionless Boltzmann equation. 
\subsection{The collisionless Boltzmann equation}
Consider particles in the phase space volume element $(p,q, t)dqdp$. If we assume that the total number of particles have not changed, a time $dt$ later, all the particles have moved to $(p, q + p/m dt, t + dt)dqdp$. Conservation of probability states that
\begin{align}
	f_N(p, q, t) dpdq = f_N(p, q + p/mdt, t + dt)dpdq.
\end{align}
which can be written as
\begin{align}
	\dpart{f}{t} + v\cdot\nabla f = 0.
\end{align}
We can extend this equation by including an external force $\vec F$ that only affects the momentum
\begin{align}
	\label{eq:collisionless_boltzmann}
	\dpart{f}{t} + v\cdot\nabla_r f + \vec F\cdot \nabla_p f = 0.
\end{align}
where we have indicated which derivatives the Del operator acts with. This equation is a good approximation to very dilute gases where intermolecular collisions occur rarely. Collisions between particles will appear as an additional term in \eqref{eq:collisionless_boltzmann}.
\subsection{The collision operator}
Suppose that the gas is dilute so that all collisions are assumed to be binary collisions, i.e. no three-particle collisions, and that the total energy and momentum is conserved. Then consider two particles $i$ and $j$ moving towards each other with velocities $\vec v$, $\vec v_1$ and relative velocity $\vec g = \vec v - \vec v_1$. In order to make the calculations easier, we will change the frame of reference. We define particle $i$ as the \textit{incident} particle and $j$ as the \textit{target} particle. After the collision, the particles will have velocities $\vec v'$ and $\vec v_1'$ with relative velocity $\vec g' = \vec v' - \vec v_1'$. If we then choose the target particle as initial frame of reference, we see that the velocity of the incident particle becomes $\tilde {\vec v} = \vec g$ and $\tilde {\vec v}' = \vec g'$.\\
Since momentum is conserved, we have that the relative velocity must remain constant $|\vec g| = |\vec g'|$ during the collision. The direction of $\vec g'$ is given by the angles $\phi$ and $\theta$ with $\hat {\vec z}$ along $\vec g$ and $\phi \in [0, 2\pi], \theta \in [0, \pi]$. We can express $\vec g'$ as
\begin{align}
	\vec g' = \vec g - 2\unit e(\unit e\cdot\vec g),
\end{align}
where $\unit e$ is some unit vector, see figure \ref{fig:relation_between_g_and_g_prime}. If we multiply by $\unit e$, we see that 
\begin{align}
	\unit e\cdot \vec g' = \unit e\cdot\left[\vec g - 2\unit{e}(\unit e\cdot\vec g)\right] = -\unit e \cdot \vec g
\end{align}
which gives the symmetric relation
\begin{align}
	\vec g = \vec g' - 2\unit e(\unit e\cdot\vec g').
\end{align}
The angle between $\vec g$ and $\vec g'$ is $\theta$, so
\begin{align}
	\vec g'\cdot \vec g = g^2\cos\theta = g^2(1 - 2\cos \chi),
\end{align}
where $\chi$ is the angle between $\unit e$ and $\vec g$ which gives the relation
\begin{align}
	\theta = \pi - 2\chi.
\end{align}
We define the solid angle element $d\Omega=\sin\theta d\theta d\phi$ about $\vec g'$ 
\begin{align}
	g d\Omega &= g\sin\theta d\theta d\phi = 2g\sin(\pi - 2\chi)d\chi d\phi\\
	&= 4g\cos\chi\sin\chi d\chi d\phi = 4\left|\unit e\cdot\vec g\right|\sin\chi d\chi d\phi\\
	&= 4\left|\unit e\cdot\vec g\right|d^2e,
\end{align}
where $d^2e = \sin\chi d\chi d\phi$ is a solid angle element about $\unit e$. In the following, we will calculate the scattering cross section which is the \textit{area} which describes the likelyhood of an incident particle being scattered by the target particle. We denote the number density as $\rho$ and find that the incident flux is $\rho g$. The rate $f_{d\Omega}'$ of scattered particles into $d\Omega$ is
\begin{align}
	\label{eq:partial_scattering_rate}
	f_{d\Omega}' = \rho g\sigma d\Omega,
\end{align}
where the proportionality constant $\sigma$ is the cross section. We might have several target particles colliding indenpendently of each other which will contribute to the scattering rate. If we have $n_t$ such particles, we obtain the total scattering rate $f_{d\Omega}$ by multiplying \eqref{eq:partial_scattering_rate} by $n_t$
\begin{align}
	f_{d\Omega} = n_t\rho g\sigma d\Omega.
\end{align}
%$\sigma d\Omega$ is independent of the choice of reference frame, so we can freely choose the center-of-mass frame to make the next calculations simpler. 
The \textit{differential} cross section $\sigma$ depends on $\vec g$ and $\vec g'$, so we denote it as $\sigma = \sigma(\vec g\rightarrow \vec g'$, whereas the \textit{total} cross section $\sigma_T$ is given by integrating over all solid angles
\begin{align}
	\sigma_T = \int d\Omega \sigma.
\end{align}
We will now look at particles with velocities in the range $d^3 v$ incident on target particles with velocity in the range $d^3 v_1$. The incident flux is $gf(v)d^3v$ and the number of target particles is $f(v_1)d^3v_1d^3x$. The rate at which particles with velocity $\vec v_1$ are scattered by particles with velocity $\vec v$ is 
\begin{align}
	f_N(\vec v)f_N(\vec v_1) g \sigma d\Omega d^3v d^3v_1 d^3 x = f_N(\vec v)f_N(\vec v_1)4\left|\unit e\cdot \vec g\right| \sigma d^2e d^3v d^3v_1 d^3 x,
\end{align}
The rate of loss is the rate of which particles in $d^3vd^3x$ are being hit by other particles. We can calulate this by integrating over all solid angles $d^2 e$ and incident velocities $d^3 v$
\begin{align}
	\text{rate of loss} = \left[\int f(\vec v)f(\vec v_1)4\left|\unit e \cdot \vec g \right|\sigma d^2 ed^3 v_1\right]d^3 vd^3 x.
\end{align}
We also have the inverse event, incident particles with velocity $\vec v'$ hitting target particles with velocity $\vec v_1'$ so that the final velocity of the target particles is $\vec v$. This is calculated with the same idea
\begin{align}
	\text{rate of gain} = \left[\int f(\vec v')f(\vec v_1')4\left|\unit e \cdot \vec g \right|\sigma d^2 ed^3 v_1\right]d^3 vd^3 x,
\end{align}
since $|\unit e\cdot \vec g| = |\unit e\cdot \vec g'|$ and $d^3 v'd^3 v_1' = d^3 vd^3 v_1$. The total change in the distribution function is given by the functional $J[f]$
\begin{align}
	J[f] &= \int d^3 v_1 d^2 e4|\unit e\cdot\vec g|\sigma[f'f_1' - ff_1]
	&= \int d^3 v_1 d\Omega g \sigma[f'f_1' - ff_1],
\end{align}
where $f = f(\vec x, \vec v, t)$ and $f_1 = f(\vec x, \vec v_1, t)$. The full Boltzmann equation is then given by
\begin{align}
	\label{eq:boltzmann_equation}
	\dpart{f}{t} + \vec v\cdot \nabla_r f + \nabla_p \vec F f = J[f].
\end{align}
\section{H-theorem}
\section{Maxwell-Boltzmann distribution as equilibrium}

