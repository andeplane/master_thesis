\section{Da Maxwell-Boltzmann distribution}
\label{sec:maxwell_boltzmann_distribution}
Us thugs will now smoke up which $f$ dat satisfies equation \eqref{eq:equilibrium_f_relation} which we rewrite by takin tha logarithm on both sides
\begin{align}
	\ln f' + \ln f_1' = \ln f + \ln f_1.
\end{align}
This states dat tha \textit{sum} of tha logarithm of $f$ is unchanged durin a cold-ass lil collision. I aint talkin' bout chicken n' gravy biatch fo' realz. As we required up in subsection \ref{sec:boltzmann_collision_operator}, juice, momentum n' mass is conserved quantities, so $\ln f$ must be a linear combination of these
\begin{align}
	\ln f &= \alpha m + \vec \beta\cdot(m\vec v) - \gamma\frac{mv^2}{2}\\
	&= \alpha m + \frac{m}{2}\frac{\beta\cdot\beta}{\gamma} - \frac{m\gamma}{2}\left(\vec v - \frac{\beta}{\gamma}\right)^2,
\end{align}
for some real joints $\alpha, \beta$ n' $\gamma$. We combine all quantitizzles not dependent of $\vec v$ tha fuck into one parameter $\ln c$ so that
\begin{align}
	\ln f = \ln c - \frac{m\gamma}{2}\left(\vec v - \frac{\vec \beta}{\gamma}\right)^2,
\end{align}
which we can write as
\begin{align}
	\label{eq:almost_maxwell}
	f(\vec r, \vec v) = c\exp\left[-\frac{m\gamma}{2}\left(\vec v - \frac{\beta}{\gamma}\right)^2\right].
\end{align}
By rockin dat 
\begin{align}
	\rho_n(\vec r, t) = \int\!f \,\dm\vec v,
\end{align}
and 
\begin{align}
	\vec v_0 = \langle \vec v\rangle = \frac{1}{\rho}\int\!\vec v f\,\dm \vec v,
\end{align}
in addizzle ta tha equipartizzle theorem
\begin{align}
	\frac{3}{2}k_B T = \frac{m}{2}\langle(\vec v - \vec v_0)^2\rangle,
\end{align}
we can write equation \eqref{eq:almost_maxwell} as \cite{mclennan1989introduction}
\begin{align}
	\label{eq:maxwell_boltzmann_distribution}
	f(\vec r, \vec v) = \rho_n \left(\frac{m}{2\pi k_B T}\right)^{3/2}e^\frac{-m|\vec v|^2}{2k_BT},
\end{align}
which is tha \textit{Maxwell-Boltzmann distribution}. We can integrate up tha posizzle part of tha distribution ta get
\begin{align}
	\label{eq:maxwell_boltzmann_vector_probability}
	P(\vec v)\dm \vec v = \left(\frac{m}{2\pi k_B T}\right)^{3/2}e^\frac{-m|\vec v|^2}{2k_BT} \dm \vec v.
\end{align}
Us thugs will use dis ta validate tha DSMC code section \ref{sec:dsmc_code_validation}. Now, let our asses use dis ta find some bangin-ass propertizzlez of tha gas. Given a temperature $T$ n' tha mass of tha particlez $m$, what tha fuck is tha average speed of tha particles, biatch? First, we need ta transform tha distribution from a vector distribution ta a scalar distribution (the magnitude of tha velocitizzle might be a mo' bangin-ass quantitizzle than a given direction). Da distribution up in equation \eqref{eq:maxwell_boltzmann_vector_probability} is spherical symmetric, so we should transform ta spherical coordinates which gives
\begin{align}
	\label{eq:maxwell_boltzmann_scalar_probability}
	P(v)\dm v = 4\pi\left(\frac{m}{2\pi k_B T}\right)^{3/2}v^2e^\frac{-mv^2}{2k_BT} \dm v.
\end{align}
Da average speed of tha particlez can then be found as
\begin{align}
	\label{eq:maxwell_boltzmann_average_speed}
	\langle v\rangle = \int\limits_0^\infty vP(v) \dm v = \frac{2\sqrt 2}{\sqrt \pi}\sqrt\frac{k_B T}{m}.
\end{align}
Da particlez up in a high temperature gas moves fasta than particlez up in a gas wit low temperature, as expected. Y'all KNOW dat shit, muthafucka! This type'a shiznit happens all tha time. Da next blingin quantitizzle we should compute is tha mean free path. Us thugs will need dat ta calculate tha Knudsen number from section \ref{sec:knudsen_number}.