\definecolor{listingsstringcolor}{rgb}{0,0.5,0}
\definecolor{listingskeywordcolor}{rgb}{0.5,0.5,0.0}
\definecolor{listingsbasiccolor}{rgb}{0.0,0.0,0.0}
% \definecolor{listingskeywordcolor}{rgb}{0.0,0.0,0.7}
\definecolor{listingsnumbercolor}{rgb}{0.0,0.0,1.0}
\definecolor{listingscommentcolor}{rgb}{0.4,0.4,0.4}
\definecolor{listingsbackgroundcolor}{rgb}{0.975,0.975,0.975}
\definecolor{listingsrulecolor}{rgb}{0.86,0.86,0.86}
\definecolor{listingsidentifiercolor}{rgb}{0.5,0.0,0.5}
% \definecolor{listingsvariablecolor}{rgb}{0.5,0.0,0.5}

\RequirePackage{iftex}

\lstset {
    language=Python,
    numbers=none,
    breaklines=true,
    tabsize=2,
    backgroundcolor=\color{listingsbackgroundcolor},
    breakatwhitespace=false,         % sets if automatic breaks should only happen at whitespace
    breaklines=true,                 % sets automatic line breaking
  %   numbers=left,                    % where to put the line-numbers; possible values are (none, left, right)
    numbersep=5pt,                   % how far the line-numbers are from the code
    frame=lrtb,                    % adds a frame around the code
    framexleftmargin=7pt,
    framexrightmargin=7pt,
    framextopmargin=7pt,
    framexbottommargin=7pt,
    xleftmargin=18pt,
    xrightmargin=18pt,
    rulecolor=\color{listingsrulecolor},
    tabsize=2,                       % sets default tabsize to 2 spaces
    literate={å}{{\aa}}1 {æ}{{\ae}}1 {ø}{{\oslash}}1,
    showstringspaces=false,
    captionpos=b,
    basicstyle=\color{listingsbasiccolor}\footnotesize\sffamily,
    keywordstyle=\color{listingskeywordcolor}\footnotesize\sffamily,
    stringstyle=\color{listingsstringcolor}\footnotesize\sffamily,
    commentstyle=\color{listingscommentcolor}\footnotesize\sffamily,
    numberstyle=\color{listingsnumbercolor}\footnotesize\sffamily,
    identifierstyle=\color{listingsidentifiercolor}\footnotesize\sffamily,
}
% \lstset{
%     literate={0}{{\textcolor{listingsnumbercolor}{0}}}{1}%
%              {1}{{\textcolor{listingsnumbercolor}{1}}}{1}%
%              {2}{{\textcolor{listingsnumbercolor}{2}}}{1}%
%              {3}{{\textcolor{listingsnumbercolor}{3}}}{1}%
%              {4}{{\textcolor{listingsnumbercolor}{4}}}{1}%
%              {5}{{\textcolor{listingsnumbercolor}{5}}}{1}%
%              {6}{{\textcolor{listingsnumbercolor}{6}}}{1}%
%              {7}{{\textcolor{listingsnumbercolor}{7}}}{1}%
%              {8}{{\textcolor{listingsnumbercolor}{8}}}{1}%
%              {9}{{\textcolor{listingsnumbercolor}{9}}}{1}%
%              {.0}{{\textcolor{listingsnumbercolor}{.0}}}{2}% Following is to ensure that only periods
%              {.1}{{\textcolor{listingsnumbercolor}{.1}}}{2}% followed by a digit are changed.
%              {.2}{{\textcolor{listingsnumbercolor}{.2}}}{2}%
%              {.3}{{\textcolor{listingsnumbercolor}{.3}}}{2}%
%              {.4}{{\textcolor{listingsnumbercolor}{.4}}}{2}%
%              {.5}{{\textcolor{listingsnumbercolor}{.5}}}{2}%
%              {.6}{{\textcolor{listingsnumbercolor}{.6}}}{2}%
%              {.7}{{\textcolor{listingsnumbercolor}{.7}}}{2}%
%              {.8}{{\textcolor{listingsnumbercolor}{.8}}}{2}%
%              {.9}{{\textcolor{listingsnumbercolor}{.9}}}{2}%
% }

\RequirePackage{calc}

%Define a reference depth. 
%You can choose either relative or absolute.
%--------------------------
\newlength{\DepthReference}
% \settodepth{\DepthReference}{g}%relative to a depth of a letter.
\setlength{\DepthReference}{0pt}%absolute value.

%Define a reference Height. 
%You can choose either relative or absolute.
%--------------------------
\newlength{\HeightReference}
% \settoheight{\HeightReference}{T}
\setlength{\HeightReference}{7pt}

\newlength{\Width}%
\newcommand{\lstinlinebox}[1]{%
  \settowidth{\Width}{\lstinline|#1|}%
  \fcolorbox{listingsrulecolor}{listingsbackgroundcolor}{%
    \raisebox{-\DepthReference}%
    {%
          \parbox[b][\HeightReference+\DepthReference][c]{\Width}{\centering \lstinline|#1|}%
    }%
  }%
}%