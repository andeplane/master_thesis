In this chapter we will go through the implementation of the MD code in the same way we did with DSMC in chapter \ref{chap:dsmc_implementation}. We discuss the stages from the beginning of the simulation and introduce implementation techniques when they appear in the process. We here assume that we want to simulate $N$ atoms in a box of physical size $L_x \times L_y \times L_z$ using the Lennard-Jones potential. Our choice of units is described in appendix \ref{app:physical_units}.

We start with section \ref{sec:md_parallelization} by briefly explaining how the code is parallelized, since this affects many of the implementation choices we have made. The steps in the actual program can then be explained. We start system initialization in section \ref{sec:md_system_init} where we describe the FCC lattice and how we choose the initial velocities of the atoms. The timestep consists of several stages that are explained in section \ref{sec:md_timestep}. In section \ref{sec:md_complex_geometries} we introduce a model of a solid allowing us to study flow in complicated geometries.