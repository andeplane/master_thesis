Molecular Dynamics is a numerical model that describes the behaviour of liquids, gases and solids at the finest scale of any classical model. We can study the dynamics of single atoms and how they interact with each other forming molecules and larger objects like advanced pore networks. The idea is simple and has been used since the time of Sir Isaac Newton in the 17th century when he formulated his laws of motion. With the knowledge of the relevant forces between the atoms, we can solve Newton's equations and calculate their dynamics. In this chapter, we will discuss how to implement an efficient parallel Molecular Dynamics algorithm and apply the Lennard Jones potential to nanoporous media. We will induce flow in such systems and study how a large surface-to-volume ratio affects liquids and gases. 