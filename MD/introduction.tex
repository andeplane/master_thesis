Molecular Dynamics is another numerical model that describes the behavior of liquids, gases and solids at the finest scale of any classical model. We can study the dynamics of single atoms and how they interact with each other forming molecules and larger objects like advanced pore networks. The idea is simple and has been used since the time of Sir Isaac Newton in the 17th century when he formulated his laws of motion. With the knowledge of the relevant forces between the atoms, we can solve Newton's equations and calculate their dynamics.

We open this chapter by a short introduction to the model in section \ref{sec:md_model} before we explain how forces are calculated with the Lennard-Jones potential in section \ref{sec:md_forces}. With computed forces, we can integrate the atoms with Newton's laws following an time integration scheme which we discuss in section \ref{sec:md_time_integration}. The last section is concerned with how we keep a constant temperature using a thermostat. Physical quantities are measured in the same way as we did in DSMC in section \ref{sec:dsmc_measuring_physical_quantities}.