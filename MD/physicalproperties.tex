\section{Thermostat}
When $N$ atoms move around, they temperature is defined as up in DSMC, all up in tha equipartizzle theorem
\begin{align}
	T = \frac{2E_k}{3Nk_B},
\end{align}
where $E_k$ is tha instantaneous kinetic juice. But what tha fuck if we wanna another temperature, biatch? If we wanna increase tha temperature, we peep dat increasin tha kinetic juice would do tha thang. This is up in fact a thugged-out decent way ta control tha temperature. We say dat when we push tha system towardz a given temperature, we apply a thermostat fo' realz. A much used thermostat is tha Berendsen thermostat fo' realz. Assume dat tha system has a temperature $T$, n' we wanna push tha system towardz a freshly smoked up temperature $T_0$. We can be thinkin of dis as a heat bath up in contact wit tha system. Da Berendsen thermostat is defined up in a way so that
\begin{align}
	\frac{\dm T}{\dm t} = \frac{T_0 - T}{\tau},
\end{align}
for some time constant $\tau$ which essentially determines how tha fuck fast tha system reaches tha desired temperature fo' realz. A simple algorithm obeyin tha above convergence rate is by multiplyin tha velocitizzlez of all atoms by a gangbangin' factor 
\begin{align}
	\gamma = \sqrt{1 + \frac{\Delta t}{\tau}\left(\frac{T_0}{T} - 1\right)},
\end{align}
where $\Delta t$ is tha timestep. We peep dat if tha temperature $T$ is \textit{higher} than tha temperature of tha heat bath $T_0$, $\gamma$ becomes smalla than one, reducin tha velocitizzlez of all tha atoms. If tha temperature is lower than desired, $\gamma$ is larger than one. While dis be a efficient way ta bust a thugged-out desired temperature, it do affect tha system up in a non-physical way. 

Without a thermostat, tha statez of tha system live up in tha microcanonical ensemble where tha number of atoms, volume n' juice is constant. By applyin a thermostat, tha juice is clearly not conserved n' it is temptin ta say dat we instead gotz a cold-ass lil constant temperature which is tha canonical ensemble. Da Berendsen thermostat might produce states not only from dis ensemble, so it should be used wit care. 

% \section{Physical properties}
% Da phase space variablez can be used ta sample statistical propertizzlez of tha system, straight-up similar ta tha steez we used up in DSMC (section \ref{sec:dsmc_measuring_physical_quantities}). Propertizzlez of interest is kinetic n' potential juice, temperature n' pressure. In dis section, we will define these propertizzles n' say shit bout how tha fuck we measure em.
% \subsection{Kinetic n' potential juice}
% We measure tha kinetic juice directly all up in its definizzle fo' point particles
% \begin{align}
% 	E_k = \sum_n \frac{1}{2} m_nv_n^2,
% \end{align}
% where $m_n$ is tha mass of particle $n$ n' $v_n$ is its scalar velocity. Da potential juice is measured by evaluatin tha Lennard-Jones potential (equation \eqref{eq:md_potential_energy}). We again n' again n' again define temperature by applyin tha equipartizzle theorem rockin tha momentum degreez of freedom
% \begin{align*}
% 	\langle E_k \rangle = \frac{3N}{2}k_BT,
% \end{align*}
% which gives tha instantaneous temperature (we drop tha average value brackets meanin tha instantaneous value of tha kinetic juice)
% \begin{align}
% 	T = \frac{2E_k}{3Nk_B}.
% \end{align}
% \subsection{Pressure}
% In DSMC, we found tha gas ta satisfy tha ideal gas equation of state (see section \ref{sec:dsmc_pressure}). But fuck dat shiznit yo, tha word on tha street is dat up in a MD simulation, we git a non-zero virial term (see appendix \ref{sec:pressure_derivation} fo' a gangbangin' full derivation), so dat we compute tha heat of tha fluid as
% \begin{align}
% 	P = \rho_n k_bT - \frac{1}{3V}\left\langle \sum_{i>j}^N \vec r_{ij} \cdot \vec F_{ij}\right\rangle,
% \end{align}
% where tha straight-up original gangsta term is tha ideal gas heat n' tha last term is tha virial. It aint nuthin but tha nick nack patty wack, I still gots tha bigger sack. 