\section{Physical properties}
The phase space variables can be used to sample statistical properties of the system, very similar to the techniques we used in DSMC (section \ref{sec:dsmc_measuring_physical_quantities}). Properties of interest are kinetic and potential energy, temperature and pressure. In this section, we will define these properties and discuss how we measure them.
\subsection{Kinetic and potential energy}
We measure the kinetic energy directly through its definition for point particles
\begin{align}
	E_k = \sum_n \frac{1}{2} m_nv_n^2,
\end{align}
where $m_n$ is the mass of particle $n$ and $v_n$ is its scalar velocity. The potential energy is measured by evaluating the Lennard-Jones potential (equation \eqref{eq:md_potential_energy}). We again define temperature by applying the equipartition theorem using the momentum degrees of freedom
\begin{align*}
	\langle E_k \rangle = \frac{3N}{2}k_BT,
\end{align*}
which gives the instantaneous temperature (we drop the average value brackets meaning the instantaneous value of the kinetic energy)
\begin{align}
	T = \frac{2E_k}{3Nk_B}.
\end{align}
\subsection{Pressure}
In DSMC, we found the gas to satisfy the ideal gas equation of state (see section \ref{sec:dsmc_pressure}). However, in an MD simulation, we get a non-zero virial term (see appendix \ref{sec:pressure_derivation} for a full derivation), so that we compute the pressure of the fluid as
\begin{align}
	P = \rho_n k_bT - \frac{1}{3V}\left\langle \sum_{i>j}^N \vec r_{ij} \cdot \vec F_{ij}\right\rangle,
\end{align}
where the first term is the ideal gas pressure and the last term is the virial. 