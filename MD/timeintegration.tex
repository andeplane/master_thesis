\section{Time integration}
\label{sec:md_time_integration}
The idea of doing a Molecular Dynamics simulation is to start from some initial state $(\vec r_0, \vec v_0)$ and compute the time evolution of atoms following Newton's equations with the defined forces. We will use this time evolution to sample statistics from states in the phase space, assuming that the time evolution will guide the system through the phase space with probabilities according to the statistical ensemble. This was the assumption of ergodicity. In a standard Molecular Dynamics simulation, we want to sample states from the microcanonical ensemble (NVE) where the energy, volume and number of atoms are conserved. Our choice of time integrator should therefore conserve energy as good as possible. A good choice is the Velocity Verlet algorithm, which is known to be both fast and conserve energy quite well\cite{frenkel2001understanding}.\\
The Velocity Verlet algorithm requires the knowledge of the positions, velocities and instantaneous accelerations of the atoms. The latter is found through the force on an atom $\vec a_i = \vec F_i/m_i$, so it requires the storage of $9N$ variables in a system with $N$ atoms. We have derived the Velocity Verlet algorithm using the Liouville formulation of time evolution in appendix \ref{app:liouville}. The integration steps are
\begin{align}
	\vec v(t + \Delta t/2) &= \vec v(t) + \frac{\vec F(t)}{m}\frac{\Delta t}{2}\\
	\vec r(t + \Delta t) &= \vec r(t) + \vec v(t + \Delta t/2)\Delta t\\
	\vec v(t + \Delta t) &= \vec v(t + \Delta t/2) + \frac{\vec F(t + \Delta t)}{m}\frac{\Delta t}{2},
\end{align}
which is done each timestep for every atom in the system. We call the first step a \textit{half kick}, because it integrates the velocity with forward Euler with half a timestep. The position is then also integrated with forward Euler with the full timestep $\Delta t$ before we do another half kick.