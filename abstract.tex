Most of the worlds currently accessible hydrocarbon resources are found in tight rocks - rocks with permeabilities in the millidarcy range and with pore sizes in the nanometer range. Tight rocks pose new scientific problems because of the small length-scales involved. Traditional oil plays are found in, for example, sand stone reservoirs with millimeter to micrometer sized pores. For such systems, standard hydrodynamics is a sufficient tool to understand, describe and predict fluid transport. In tight rocks, however, typical pore sizes are in the range of tens to hundreds of nanometers. At this scale, the mean free path - the average distance a particle moves between collisions - may become comparable to the characteristic sizes of the porous medium, and the standard assumption that the fluid can be described as a continuum no longer holds. The mean free path of dilute gases are often tens of nanometers, or higher. 

The gas from tight rocks, like shales, is stored inside closed pore networks or adsorbed onto organic matter. In order to extract adequate levels of gas from such rocks, we generate fractures to increase the permeability of the rock. The gas flows from small pore networks with diameters down to a few nanometers. The rate at which the gas flows through such networks is proportional to the permeability of the material - a result known as Darcy's law. Dilute gases in nanoporous media have a non-zero slip velocity which can cause an increase of permeability of a factor 50 compared to what continuum theory predicts. This is an effect known as the Klinkenberg effect. In order to be able to increase gas production rates in a safe way, we need to understand the physics at this scale. This requires models that are valid at these length scales.

In this thesis, we implement two numerical particle models. The first is called Molecular Dynamics. It describes the motion of atoms by using parameterized potentials to compute forces between them. The second model, Direct Simulation Monte Carlo, uses the principles of statistical mechanics allowing larger systems to be studied. Both implementations support arbitrary geometries, and show promising scaling efficiency on massive parallel machines. We use these models to study the Klinkenberg effect and confirm that the Knudsen permeability correction correctly predicts the permeability for systems with pore sizes down to a few nanometers. We also analyze more complicated geometries, and argue that a stochastic version of the Knudsen correction is needed for geometries without a well defined Knudsen number. 

Highly efficient custom 3D visualization tools are developed using modern OpenGL techniques such as instanced geometry shaders, billboards and the marching cubes algorithm. Systems with tens of millions of particles can be rendered realtime with decent frame rate, allowing us to study larger systems than what can be done with already existing free software. All software developed during this thesis can serve as tools for further study in the field.