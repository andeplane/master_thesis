Whenever scientists big-ass up a experiment up in tha lab or on a cold-ass lil computer, they need ta study tha shiznit outcome of tha experiment. This could be tha positionz of some particles, tha temperature of a gas or maybe tha color of a gangbangin' fluid. Y'all KNOW dat shit, muthafucka! Once our crazy asses have dis shiznit, we need ta represent it up in a way dat is useful fo' understandin tha thangs up in dis biatch. We often put numbers up in a table or deal dem as a graph. This be a mad bangin ways ta peep data n' we can learn a shitload by seein how tha fuck two or mo' variablez is dependent of each other n' shit. Well shiiiit, it is convenient ta introduce a general term, \textit{visualization}, which is just a visual representation of shiznit (maybe not \textit{any} visual representation. I aint talkin' bout chicken n' gravy biatch. Numbers up in a table might be excluded.) One could say dat a ghetto map be a representation of tha geometrical shiznit describin how tha fuck tha continents is connected. Y'all KNOW dat shit, muthafucka! This type'a shiznit happens all tha time. Da graph of some data be a relation between two or mo' variables. Da output result from any computer simulation or experiment is shiznit dat can be visualized up in some way.

There already exists a shitload of software dat can be used ta visualize data up in different ways. For particle simulations one can fo' example use \textit{VMD}\footnote{\url{http://www.ks.uiuc.edu/Research/vmd/}} or \textit{Ovito}\footnote{\url{http://www.ovito.org/}}. Both of these is pimped out tools allowin our asses ta visualize tha time trajectoriez of atoms or particles. There is two drawbacks dat have motivated tha lyricist ta write a visualization tool from scratch.

In both VMD n' Ovito, tha way we navigate wit tha camera is inspired by other 3d software like 3dz Max, Maya n' Blender where tha camera is pointin towardz a point whereas tha mouse controls tha posizzle of tha camera on tha surface of a sphere centered up in point. If we wanna follow tha movement of a single particle, dis way of controllin tha camera is inconvenient. Us thugs wanna control tha camera up in tha way a space shizzle is controlled up in a game, like up in a gangbangin' first thug blasta game yo. Here tha mouse controls tha direction tha camera is lookin n' tha camera posizzle is controlled by tha keyboard.

In addition, there be performizzle drawbacks wit both programs where our crazy asses have noticed dat medium sized datasets (number of particlez $\approx 1\e{6}$) lead ta a gangbangin' frame rate dat make it straight-up hard as fuck ta navigate round up in tha system fo' realz. Also, tha geometry of tha DSMC code cannot be visualized up in VMD or Ovito since these is particle visualizers only n' tha geometry is represented as a scalar field (the voxels from section \ref{sec:dsmc_complex_geometries}). To be able ta solve these problems, our crazy asses have implemented our own visualizer rockin OpenGL. In dis chapter, we first give a funky-ass brief introduction ta OpenGL up in section \ref{sec:opengl} explainin its basic ideas. We go all up in tha conceptz of vertices, primitives n' how tha fuck flavas n' textures is linearly interpolated from tha jointz of tha vertices. Vertex Buffer Objects is briefly explained up in section \ref{sec:opengl_vbo} before we go all up in tha sequence of shadaz up in tha renderin pipeline up in section \ref{sec:opengl_rendering_pipeline}. Note dat dis aint a cold-ass lil complete introduction of OpenGL. Only tha basic concepts dat is required ta KNOW how tha fuck our crazy asses have made tha visualization tools is covered.