With our newly acquired knowledge bout OpenGL n' hustled how tha fuck we can use tha API ta render objects on tha screen, our crazy asses have every last muthafuckin thang we need ta pimp our own visualization tools dat can handle datasets from both MD n' DSMC fo' realz. As we now should be well aware of, tha state of a system wit $N$ particlez is busted lyrics bout by tha $3N$ particle positions n' tha $3N$ velocitizzle components, n' you can put dat on yo' toast. If we save dis shiznit every last muthafuckin timestep of a simulation, we can use it ta render a time series, a animation of tha trajectoriez of all tha particles. Us thugs will render tha particlez as spheres yo, but is goin ta cheat a funky-ass bit fo' realz. An actual sphere rendered up in OpenGL would need ta be composed of nuff trianglez formin tha spherical shape. To be able ta render a smooth sphere, we would need mo' than 100 trianglez \textit{per sphere} as we will peep up in section \ref{sec:vis_billboards}. Us thugs will apply a trick used up in computer game fo' years. Instead of renderin spheres, we use suttin' called \textit{billboards}, which be a rectangle wit a \textit{image} (a texture) of a sphere, always pointin towardz tha camera. In section \ref{sec:vis_billboards}, we explain how tha fuck we effectively can create n' render billboardz wit tha geometry shader on tha GPU. If tha particle system has periodic symmetry (both MD n' DSMC use periodic boundary conditions), we can also use tha geometry shader ta render copiez of tha system, makin tha illusion dat tha system is larger than it straight-up is.

Us dudes did a performizzle test ta peep tha amount of particlez it is possible ta visualize wit dis method. Y'all KNOW dat shit, muthafucka! Da benchmark measure is tha number of frames per second tha program can manage ta render n' shit. This is done up in section \ref{sec:vis_benchmark}. When we visualize a thugged-out dataset from a DSMC simulation, we should, up in addizzle ta tha particle positions, render tha surface geometry (which, as we remember from section \ref{sec:dsmc_complex_geometries}, be a voxelized scalar field). Us thugs will then be able ta peep tha surface tha particlez collide wit which will make it easier ta KNOW they behavior. Shiiit, dis aint no joke. In section \ref{sec:marching_cubes}, our phat asses say shit bout tha so-called \textit{marchin cubes} algorithm, which allows our asses ta create a set of renderable trianglez from tha iso-surface of a scalar field (the points where tha scalar field joints intersect some value).