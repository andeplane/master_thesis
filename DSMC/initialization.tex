\section{System initialization}
\label{sec:dsmc_implementation_initialization}
Now that we know how the geometry is represented, we are ready to discuss the DSMC simulator itself. When we want to run a simulation in a given geometry, we need to decide a few things. We need to specify
\begin{itemize}
	\item the density ($\rho_n)$,
	\item the temperature ($T_0$),
	\item the physical system size ($L_x, L_y, L_z$), and
	\item the number of atoms each simulated particle represents ($N_\text{eff}$).
\end{itemize}
These are needed input parameters that will affect the initialization process. The first step when the program starts is to load the geometry from disk. We then create the collision cells, before we add all the particles to the system. 
\subsection{The geometry}
We remember that everything we need to know about the geometry are the voxel matrix of size $N_xN_yN_z$, a normal vector and two tangent vectors per voxel. This data is saved in a class \classname{Grid} where the geometry data is saved in four variables and lookup functions as shown in listing
\begin{lstlisting}[caption=A Grid class example. This class contains everything we need to know about the geometry., label=lst:dsmc_class_grid]
typedef enum {
    voxel_type_empty = 0,
    voxel_type_wall = 1,
    voxel_type_boundary = 2
} voxel_type;

class Grid
{
private:
	int nx, ny, nz; // Number of voxels
	vector<unsigned char> voxels;
	vector<Vector3> normals;
	vector<Vector3> tangents1;
	vector<Vector3> tangents2;
public:
	unsigned char &get_voxel(const int &i, const int &j, const int &k) {
    	return voxels[i*ny*nz + j*nz + k];
	}

	unsigned char &get_voxel(Vector3 &position) {
		int i = nx * (position.x / system_size.x);
		int j = ny * (position.y / system_size.y);
		int k = nz * (position.z / system_size.z);

    	return get_voxel(i,j,k);
	}

	// The other three are similar
}
\end{lstlisting}
The first part defines the different values a voxel can have, empty, wall and boundary. The voxel values and associated surface vectors are stored in \classname{std::vector} objects in a linear form (an array with one index). This means that given the three voxel coordinates $(i,j,k)$, they are mapped onto one index as shown in the code example.
\subsection{Collision cells}
As we remember from section \ref{sec:dsmc_collisions_model}, we will perform collision between particles that are in the same collision cell only. We haven't really talked about what such a collision cell \textit{is} yet, except that its size should be smaller than one third of the mean free path. The simplest way to create these cells is to just divide the total system volume into smaller boxes of equal size. Then each of these cells should have control over which particles that are inside the volume it represents (or owns if you like). Then each timestep, a number of collisions is performed in each cell. This number was found to be (equation \eqref{eq:dsmc_number_of_collisions})
\begin{align}
	\nonumber
	M_\text{coll} = \frac{N_c(N_c-1)N_\text{eff}\pi d^2\langle v_r \rangle \Delta t}{2 V_c}.
\end{align}
We see that one of the factors is the cell volume $V_c$. But some of the cells may have large parts of their volume unavailable for fluids, they have a \textit{local porosity}. This is okay, we just need to loop through all of the voxels within a cell and compute the local porosity. An example of how this can be done is shown in listing \ref{lst:dsmc_initialize_cells}.
\begin{lstlisting}[caption=Example code showing how to find porosity and volume of the collision cells., label=lst:dsmc_initialize_cells]
void create_cells() {
    for(int k=0;k<grid.nz;k++) {
        int c_z = float(k)/grid->nz*cells_z;
        for(int j=0;j<grid.ny;j++) {
            int c_y = float(j)/grid->ny*cells_y;

            for(int i=0;i<grid.nx;i++) {
                int c_x = float(i)/grid->nx*cells_x;
                // Find the one-dimensional cell index 
                int cell_index = cell_index_from_ijk(c_x,c_y,c_z);
                // Count both the total number of voxels
                // and the number of empty voxels
                Cell &cell = cells.at(cell_index);
                cell.total_voxels++;
                cell.empty_voxels += world_grid->get_voxel(i,j,k)<voxel_type_wall;
            }
        }
    }

    for(int i=0; i<cells.size(); i++) {
    	Cell &cell = cells.at(i);
    	double cell_porosity = cell.empty_voxels / cell.total_voxels;
    	cell.porosity = cell_porosity; // Set porosity
    	cell.volume *= cell_porosity;  // Update volume
    }
}
\end{lstlisting}
\subsection{Particles}
Assuming that we have created the grid object, filled in the voxel array and created all the collision cells, we are ready to create all the particles. As mentioned earlier in this section, we need to specify the system size ($L_x, L_y, L_z$). The total system volume is then of course found as $V_\text{system} = L_xL_yL_z$. But we are going to study a system with a given porosity, the number of volume available to the fluid divided by the total volume. The porosity was found in equation \eqref{eq:dsmc_geometry_porosity} by counting all the empty voxels. The volume available for fluid is then $V = \phi V_\text{system} = \phi L_xL_yL_z$, which combined with the density $\rho_n$ gives us the total number of \textit{atoms} in the system 
\begin{align}
	N_\text{atoms} = \rho_n V.
\end{align}
But we are going to create $M$ particles, each representing $N_\text{eff}$ real atoms, so the total number of particles in our system becomes
\begin{align}
	M = \frac{N_\text{atoms}}{N_\text{eff}} = \frac{\rho_n V}{N_\text{eff}}.
\end{align}
Each of these particles is assigned a random position in the physical space. If the particle is placed inside a wall, then we find a new, random position until it is safely placed inside the available pore space. Each particle gets a velocity according to the input temperature $T_0$ where each velocity component is a normal distribution with standard deviation $\sigma_v = \sqrt{k_BT_0/m}$. Once we have found a position for the particle, we need to add it to the collision cell corresponding to its position. An example code showing how this is done is found in listing \ref{lst:dsmc_initialize_particles}.
\begin{lstlisting}[caption=Particle initialization., label=lst:dsmc_initialize_particles]
void initialize_particles() {
	double system_volume = system_size.x * system_size.y * system_size.z;
	int num_particles = density*volume*porosity / num_atoms_per_particle;
	double velocity_standard_deviation = sqrt(boltzmann_constant*temperature / mass);
	for(int index=0; index<num_particles; index++) {
		// First, assign velocities from the Maxwell-Boltzmann distribution
		velocities.at(index).x = rnd.nextGauss() * velocity_standard_deviation;
		velocities.at(index).y = rnd.nextGauss() * velocity_standard_deviation;
		velocities.at(index).z = rnd.nextGauss() * velocity_standard_deviation;

		find_position(index);

		// Find the collision cell and add the particle
		int cell_index = cell_index_from_position( positions.at(index) );
		Cell &cell = cells.at(cell_index);
        cell.add_particle(index);
	}

	void find_position(const int &index) {
		// Assume that the particle is inside the wall
		// until proven otherwise
	    bool is_inside_wall = true;
	    Vector3 &position = positions.at(index);
	    
	    while(is_inside_wall) {
	        position.x = system_size.x*rnd->next_double();
	        position.y = system_size.y*rnd->next_double();
	        position.z = system_size.z*rnd->next_double();

	        // Check if this voxel has value equal to voxel_type_wall or voxel_type_surface
	        is_inside_wall = world_grid->get_voxel(position)>=voxel_type_wall;
	    }
	}
}
\end{lstlisting}
This method will uniformly distribute all the particles in the available pore space. We are now ready to perform the timesteps.