\section{Complex geometries}
\label{sec:dsmc_complex_geometries}
All the surface interaction models from section \ref{sec:surface_interactions} use the surface normal and tangent vectors to calculate the reflected velocities. These vectors are easy to determine if the system consists of two parallel plates in the xy-plane, or any other mathematically well described geometry. Such systems are interesting as validation test cases, but most real world materials have a more complex geometry without any simple mathematical description. A very much used representation of such geometries is a triangle mesh in which the surface consists of many connected triangles. The triangles have a well defined normal vector and tangent plane which is easy to calculate. With this method, collision detection is done by checking intersection with each triangle and is rather computationally expensive. In this thesis, I have chosen another approach by representing the system as a large, binary three-dimensional matrix consisting of voxels, each having the value \textit{filled} or \textit{empty}. With this model, collision detection is done by a quick memory lookup to check if the voxel corresponding to the position of a particle is filled or not. In this section we discuss how to create such a matrix, how to identify the surface points and how to calculate the vectors describing the surface geometry.

\subsection{Binary representation - voxels}
\label{sec:dsmc_binary_representation}
With this method, any system geometry is fully described by a three dimensional matrix with dimensions $m\times n\times l$. Each matrix element represents a voxel in the physical space, and can take values 0 or 1. A value of one means that the voxel is filled, whereas zero means the voxel is empty. A particle cannot be inside a filled voxel, which is how we do surface collision detection. 
\subsection{Collision detection}
We define a collision event as when a particle has moved into a filled voxel during the timestep $\Delta t$. This has to be checked for every particle each timestep. This is rather easy to do and is best illustrated by a code example \todo{Use Vector3-syntax on vectors}
\begin{lstlisting}
bool did_collide(double *position) {
	int voxel_index_i = position[0] / system_length[0] * num_voxels[0];
	int voxel_index_j = position[1] / system_length[1] * num_voxels[1];
	int voxel_index_k = position[2] / system_length[2] * num_voxels[2];

	// The world matrix is a binary matrix
	return world_matrix[voxel_index_i, voxel_index_j, voxel_index_k];
}
\end{lstlisting}
This is just a quick memory lookup. But since a particle during the timestep in principle could have moved through many voxels into the wall, the expensive part of the algorithm is to identify the voxel that is the \textit{first} surface voxel the particle did hit (which is the voxel the particle actually collided with). We need to precalculate all the surface voxels so they are already marked in the matrix during runtime.
\subsection{Identifying the surface voxels}
Given the binary matrix, we have identified every solid part of the system. The voxels inside a wall that are not part of the surface all have neighbouring voxels that are also marked as walls. We \textit{define} the surface as the filled voxels that have less than 26 neighbouring filled voxels. The algorithm could be implemented like this (one would also have to take care of the periodic boundary conditions, but that is not important to illustrate the idea):
\begin{lstlisting}
bool is_surface(short ***world_matrix, int voxel_index_i, int voxel_index_j, int voxel_index_k) {
	for(int i=-1;i<1;i++) {
    	for(int j=-1;j<1;j++) {
			for(int k=-1;k<1;k++) {
				// Skip self
				if(i == j == k == 0) continue; 
                if(world_matrix[voxel_index_i + i][voxel_index_j + j][voxel_index_k + k] == 0) {
                	// This neighbour is empty
                	return true;
                }
            }
        }
    }

    return false;
}
\end{lstlisting}
This has to be done for every voxel in the system, but only once per system. 
\subsection{Calculating normal and tangent vectors}
The last surface properties we need to calculate are the normal and tangent vectors. This could in principle be done by using marching cubes \cite{article:marching_cubes_original} or a similar technique. However, in this thesis, I have chosen to develop a new way of describing the surface vectors. A cube consisting of 9 voxels has a geometric center $\vec r_{gc}$, plus a center of mass $\vec r_{cm}$ which can be defined through the values, the mass, of the voxels
\begin{align}
	\vec r_{cm} = \sum_i\sum_j \vec r_{ij}m_{ij},
\end{align}
where $m_{ij} \in \{1,0\}$. We \textit{define} the normal vector to be 
\begin{align}
	\vec n = 
\end{align}
\subsection{Scaleability}