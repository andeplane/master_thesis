\section{Surface interactions}
The effects of surface interactions become significant as the pore sizes decrease. For very small pores, the number of atoms near the surface is comparable with total number of atoms. In MD, these effects are already taken care of through the atomic forces, but in DSMC, we need a surface interaction model. We investigate four different models in this section. The main property of these models is to perform a statistically correct energy and momentum transfer between the wall and the colliding particles. In DSMC, we are only interested in the macroscopic details of these collisions. There are two important parameters that are used, the momentum and thermal accomodation coefficients.
\subsection{Accomodation coefficients}
When a particle hits a wall with energy $E_i$, some of the energy might be transferred to the wall resulting in an energy change $\Delta E$. On average, we can define the \textit{thermal accomodation coefficient} 
\begin{align}
	\sigma_T = \frac{E_i - E_r}{E_i - E_w},
\end{align}
where $E_i$ is the energy of the incoming particles, $E_r$ is the energy of the outging molecules and $E_w$ is the energy corresponding to the surface temperature $T_w$. A thermal accomodation coefficient equal to zero would mean that there is no energy exchange, and we will get the specular wall model described below. $\sigma_T=1$ on the other hand means that all the reflected particles have energies corresponding to the surface temperature. This is the thermal wall, and there is no correlation between the incoming and outgoing velocities. More intricate models make use of other values of the accomodation coefficients so the particles \textit{remember} their incoming velocities.

\subsection{Specular wall}
The specular wall behaves just like a classical mirror. The colliding objects are reflected so that the normal component of the velocity is reversed while the tangential components remain unchanged. There is no exchange of energy with the wall. This model isn't very interesting, but worth mentioning.

\subsection{Thermal wall}
If we instead think of the wall as an object with a given temperature $T_w$, we can imagine that the particles go into the wall, collide with the wall atoms as a random walk, and return with no correlation with the incoming velocity. We can then choose a new, random velocity vector from a distribution so that the gas temperature converges to the wall temperature. The distribution has to reflect that faster particles collide more often with the surface. We can derive the distribution by looking at the velocity distribution of particles going through an imaginary wall during a time $\Delta t$.
\subsubsection{An imaginary wall}
Imagine an area $A$ randomly placed inside an isotropic gas, and look at the number of particles with velocities in the range $(v, v+dv)$ with direction $(\theta, \theta + d\theta)$ and $(\phi, \phi + d\phi)$. If we assume that the gas is isotropic, the number of particles having velocities within these ranges is 
\begin{align*}
	\frac{N}{V}f(v)dv \frac{d\Omega}{4\pi} = nf(v)dv \frac{d\Omega}{4\pi},
\end{align*}
where $d\Omega=\sin\theta d\theta d\phi$. During a time $\Delta t$ small enough that there are no collisions between particles, the volume of the prism is $v\Delta t A$, and the number of collision per time per area is 
\begin{align*}
	\frac{n}{4\pi} v f(v)dv \sin\theta d\theta d\phi.
\end{align*}
If we only look at the particles passing through from one side of the wall, we integrate over the angles and get the number density
\begin{align*}
	d\nu(v) &= \frac{n}{4\pi} v f(v)dv \int_0^{2\pi}\int_0^{\pi/2}d\theta d\phi \sin\theta\\
	&= \frac{n}{4} v f(v) dv.
\end{align*}
The total number of particles that passes through the wall is
\begin{align*}
	\nu = \int_0^\infty \frac{n}{4} v f(v) dv = \frac{n}{4} \langle v \rangle.
\end{align*}
We get the normalized distribution by using
\begin{align*}
	f_e(v) = \frac{d\nu}{\nu} = \frac{v}{\langle v \rangle} f(v) dv = \frac{1}{2} \left(\frac{m}{kT}\right)^2 v^3 e^{-mv^2/2kT}dv,
\end{align*}
which is very similar to the Maxwell distribution but notice that it gives slightly faster particles.

\subsection{Maxwell scattering}
\begin{align}
	R(\vec v'\rightarrow \vec v; x) &= (1-\sigma_v)\delta(\vec v' - \vec v + 2\vec nv_n) + \frac{2\sigma_v\beta_w^4}{\pi}\exp(-\beta_w^2v^2)
\end{align}

\subsection{The Cercignani-Lampis model}
Maxwell's scattering kernel was widely used until the 1960s, before more detailed models were developed. As of today, the most popular one is the Cercignani-Lampis model \cite{book:micro_and_nano}. The kernel is given as
\begin{align}
	\nonumber
	R(\vec v'\rightarrow \vec v; x) &= \frac{2\sigma_n\sigma_t(2-\sigma_t)\beta_w^4}{\pi}\\
	\nonumber
	&\times\exp\Big(-\beta_w^2\frac{v_n^2 + (1-\sigma_n)(v_n')^2}{\sigma_n} - \beta_w^2\frac{(v_t - (1 - \sigma_t)v_t')^2}{\sigma_t(2 - \sigma_t)}\Big)\\
	&\times I_0\Big(\beta_w^2\frac{2\sqrt{1 - \sigma_t}v_nv_n'}{\sigma_n}\Big),
\end{align}
where $v_n$ and $v_t$ again are the normal and tangential components of the velocities and $I_0$ is the zeroth-order modified Bessel function of the first kind. 

\subsection{The random surface model}