\section{Pressure driven flows}
In order to induce flow in a system, it is common to apply a pressure gradient. A pressure gradient means that there acts a nonzero net force on any volume element $dV$ in the system. In continuum models like the NSE, the pressure (and hence pressure gradients) is incorporated as boundary conditions where pressure is specified at given points. In particle models like DSMC, pressure must be defined through statistical mechanics and is not really a required property.

\subsection{Gravity}
We look at a volume element of size $\Delta x\Delta y\Delta z$ in a channel with a continuous fluid and a pressure gradient in the $x$-direction, see figure \ref{fig:pressure_gravity_equivalent}. The force acting on the volume element is
\begin{align}
	F = P_1\Delta y\Delta z - P_2\Delta y\Delta z = \Delta y\Delta z\Delta P,
\end{align}
where $\Delta P = P_1 - P_2$. We aim to find a constant force $F=ma$ being equivalent to that of the pressure difference. With an applied gravity $g$, the force is then
\begin{align}
	F = mg = \rho_m \Delta V g = \Delta y\Delta z\Delta P = \Delta V \frac{P}{\Delta x},
\end{align}
which gives the relation
\begin{align}
	g = \frac{P}{\rho_m\Delta x}.
\end{align}
In simple systems like a tube, the steady state velocity profiles for both pressure models should be comparable. However, for disordered systems with regions that can \textit{trap} particles (see figure \ref{fig:gravity_trap}), we will see that the gravity model has some effects that will affect the fluid flow in a non-physical way. 