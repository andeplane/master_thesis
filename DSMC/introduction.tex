Now we have the theoretical foundation we need to develop the first numerical model we will use to study flow in nanoporous media. It is called Direct Simulation Monte Carlo (DSMC) and is a stochastic particle model that has showed incredible predictive power for flow in the high Knudsen number regime. The model was developed by G. A. Bird in 1976 and was quicly picked up by engineers working in the field of aerospace. In the upper atmosphere (\unit{100}{\kilo\meter}), the mean free path of air is several meters. This gives a Knudsen number of order unity since the size of the nose of a space shuttle is of order meter \cite{alexander1997direct}. In the later years, the method has been widely used to study microflows which is, of course, our main concern.\\
We start the chapter by introducing the model by discussing the main properties of the model, and what it aims to do. There are two crucial parts of the model, collisions between particles which is discussed in section \ref{sec:dsmc_collisions_model}, and how the particles interact with the surface which is covered in section \ref{sec:surface_interactions}. Another important subject is of course how we measure physical quantities like temperature and energy. This is described in section \ref{sec:dsmc_measuring_physical_quantities}. The DSMC model does not operate with forces between particles in the classical sense, so the virial in the pressure cannot be computed directly. In section \ref{sec:dsmc_pressure} we have a longer discussion about the pressure and argue that a DSMC as actually satisifies the ideal gas equation of state. We also derive a relationship between a given pressure difference $\Delta P$ and a constant force allowing us to induce flow in the system without having to have large gradients in the density or temperature (which is necessary since we work with an ideal gas). We then have a brief comment about the numerical stability and how the choice of timestep and collision cell size introduce errors in transport coefficients. We complete the chapter by discussing how we determine whether or not a system has reached a steady state in section \ref{sec:dsmc_steady_state}. The implementation of all these steps are explained in detail in chapter \ref{chap:dsmc_implementation}.