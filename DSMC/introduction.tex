We now have the theoretical foundation we need to develop the first numerical model we will use to study flow in nanoporous media. It is called Direct Simulation Monte Carlo (DSMC), and is a stochastic particle model that has showed incredible predictive power for flow in the high Knudsen number regime. The model was developed by G. A. Bird in 1976 and was quickly picked up by engineers working in the field of aerospace. In the upper atmosphere (\unit{100}{\kilo\meter}), the mean free path of air is several meters. For space shuttles, this gives a Knudsen number of order unity since the size of its nose is of order meter \cite{alexander1997direct}. In the later years, the method has been widely used to study microflows which is our main concern in this thesis. In 1992, the model was proved to converge towards a solution of the Boltzmann equation (equation \eqref{eq:boltzmann_equation}) in the limit where the timestep $\Delta t\rightarrow 0$ and the number of particles $M\rightarrow \infty$ \cite{wagner1992convergence}.

We start the chapter by introducing the model and its basic philosophy. The model has two two crucial parts, collisions between particles which is discussed in section \ref{sec:dsmc_collisions_model}, and how the particles interact with the surface. The latter is covered in section \ref{sec:surface_interactions}. Another important subject is of course how we measure physical quantities like temperature and energy. This is described in section \ref{sec:dsmc_measuring_physical_quantities}. In section \ref{sec:dsmc_pressure} we have a longer discussion about the pressure and argue that a DSMC gas actually satisfies the ideal gas equation of state. We also derive a relationship between a given pressure difference $\Delta P$ and a constant force allowing us to induce flow in the system without needing large gradients in the density or temperature. We then have a brief comment about the numerical stability and how the timestep and collision cell size introduce errors in transport coefficients. We complete the chapter by discussing how we determine whether or not a system has reached a steady state in section \ref{sec:dsmc_steady_state}. The implementation of all these steps are explained in detail in chapter \ref{chap:dsmc_implementation}.