\section{Equation of state}
\label{sec:dsmc_eos}
The free \textit{modules} in a DSMC simulation are the collision operator $\mathcal C$ and the move operator $\mathcal M$ which fully (stochastically) determines the time evolution of the system. For hard sphere particles, the pressure may be defined in a similar way as for MD (equation \eqref{eq:pressure_in_md})
\begin{align}
	P = \rho_nkT + {1\over tV}\sum_{all collisions} m\Delta \vec v_{ij}\cdot \vec r_{ij},
\end{align}
where $\Delta \vec v_{ij}$ is the change of velocity of one of the particles during a collision and $\vec r_{ij}$ is the distance between these two particles\cite{garcia1997direct}. If we choose the hard sphere collision model as described in the previous section, there is no correlation between the change in velocity $\Delta \vec v_{ij}$ and the displacement vector $\vec r_{ij}$
\begin{align}
	\langle \Delta \vec v_{ij}\cdot \vec r_{ij}\rangle = 0,
\end{align}
so the expression for the pressure is reduced to that of ideal gas
\begin{align}
	P = \rho_n kT.
\end{align}
Since the main focus of this thesis is to study dilute gases where the ideal gas is a good approximation, this collision model is sufficient enough. For dense gases, it is possible to apply collision models that yields other equations of state.
\subsection{Non-ideal gas corrections}