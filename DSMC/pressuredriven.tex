\subsection{Measuring permeability}
\label{sec:permeability_acceleration_driven}
\todo{Here also, use the $\rho g$ version of darcy's law}
The permeability was defined in section \ref{sec:permeability_dsmc} through Darcy's law (equation \eqref{eq:darcy_1})
\begin{align}
	k = \frac{Q\mu L}{A \Delta P},
\end{align}
but we did not know how to correctly evaluate the pressure since $x=0$ is the same point as $x=L$ in a periodic system with length $L$. We can define the pressure at $x=0$ through the ideal gas law, and due to the acceleration $g$ there is an implied pressure at $x=L$ we can find by using equation \eqref{eq:acceleration_to_pressure_difference}
\begin{align}
	P_0 &= \rho_n(x=0)kT\\
	P_L &= P_0 - \Delta P \\
	&= \rho_n(x=0)kT - \bar{\rho}_m g L,
\end{align}
where $\bar{\rho}_m$ is the average density in the system and we have used that $\Delta x=L$.\\
We still need to figure out how to measure the volumetric flow rate $Q$. The volumetric flow rate tells us how much volume that passes through a surface per time. The simplest way to measure $Q$ is to count how many particles that have passed through a surface and multiply that number with the volume per particle $\rho_n^{-1}$. Since we apply periodic boundary conditions on particles that fly out of the system, we just have to keep track of how many, this is illustrated in listing \ref{lst:dsmc_apply_periodic_boundary_conditions}.
\begin{lstlisting}[caption=Example of how to apply periodic boundary conditions and at the same time calculate number flux., label=lst:dsmc_apply_periodic_boundary_conditions]
void apply_periodic_boundary_conditions(vector<double>&position, vector<long> &count_periodic, vector<double> &system_length)
{
    if(position[0] >= system_length[0]) { 
    	position[0] -= system_length[0]; count_periodic[0]++; 
    }
    else if(position[0] < 0) { 
    	position[0] += system_length[1]; count_periodic[0]--; 
    }

    if(position[1] >= system_length[1]) { 
    	position[1] -= system_length[1]; count_periodic[1]++;
    }
    else if(position[1] < 0) {
    	position[1] += system_length[1]; count_periodic[1]--; 
    }

    if(position[2] >= system_length[2]) { 
    	position[2] -= system_length[2]; count_periodic[2]++; 
    }
    else if(position[2] < 0) {
    	position[2] += system_length[2]; count_periodic[2]--; 
    }
}
\end{lstlisting}
Now that we have the number flux, we can easily calculate the permeability as shown in listing 
\begin{lstlisting}[caption=Calculation of permeability\, assuming that flow is in the $z$-direction., label=lst:dsmc_permeability]
double calculate_permeability(double volume, long num_particles, double viscosity, vector<long> &count_periodic, vector<double> &system_length, double mass_density, double acceleration)
{	
    double volume_per_particle = volume / num_particles;

    // We assuming that the flow is in z-direction
    double volume_flow_rate = count_periodic[2]*volume_per_molecule;
    double area = system_length[0]*system_length[1];

    double pressure_0 = system->density*system->temperature;
    double pressure_L = pressure_in_reservoir_0 - mass_density*acceleration*system_length[2];
    double permeability = 2*pressure_0*volume_flow_rate*system_length[2]*viscosity / (area * (pressure_0*pressure_0 - pressure_L*pressure_L));

    return permeability;
}
\end{lstlisting}