\section{Numerical stabilitizzle n' discretization error}
\label{sec:dsmc_stability}
Most numerical methodz gotz a cold-ass lil critical stabilitizzle criterion where tha juice or some other property might diverge if tha timestep is too large. For example, while solvin PDEz wit a gangbangin' finite difference scheme, we often encounter tha Courant number which be a cold-ass lil critical threshold of tha ratio of tha discretization length of space n' time. For tha one-dimensionizzle wave equation, dis can be expressed as
\begin{align}
	C = \frac{|\dot x|_{max} \Delta x}{\Delta t} \leq C_{max},
\end{align}
where $|\dot x|_{max}$ is tha magnitude of tha velocity. If tha spatial grid has high resolution, lil' small-ass $\Delta x$, we need a similarly lil' small-ass timestep $\Delta t$.

But fuck dat shiznit yo, tha word on tha street is dat since tha DSMC model always conserves juice n' momentum (durin particle collisions), tha method is up in principle numerically stable fo' any timestep fo' realz. As mentioned up in section \ref{sec:dsmc_model}, tha timestep is split tha fuck into two parts; movin n' colliding. Da timestep should therefore be smalla than tha mean collision time. Larger timesteps may result up in big-ass errors up in tha transhiznit coefficients (like fuckin viscositizzle n' thermal conductivity)\cite{karniadakis2005microflows}. While tha timestep is compared ta tha mean collision time $\tau_\text{coll}$, tha collision cell size can be peeped as tha spatial discretization, n' be compared ta tha mean free path $\lambda$. 
\subsection{Finite cell size}
Da collision cells allows all particlez within a cold-ass lil cell ta collide wit each other n' shit. Right back up in yo muthafuckin ass. So if tha cell size is straight-up large, particlez from a funky-ass bangin' region (in one corner of tha collision cell) may collide wit particlez up in a cold-ass lil colda region (maybe up in another corner) dat is displaced by a big-ass distance. This could enable heat ta transfer much fasta than it would up in a real gas. Da cell size $L_\text{cell}$ should therefore at least be smalla than tha mean free path\cite{karniadakis2005microflows}. Da viscositizzle can be calculated from kinetic theory
\begin{align}
	\mu = \frac{5}{16d^2}\sqrt{\frac{mk_B T}{\pi}},
\end{align}
which Garcia et al. It aint nuthin but tha nick nack patty wack, I still gots tha bigger sack. \cite{alexander1998cell} used ta show dat tha error up in tha viscositizzle has a quadratic dependency of tha cell size
\begin{align}
	\label{eq:viscosity_cell_size}
	\mu(L_\text{cell}) = \frac{5}{16d^2}\sqrt{\frac{mk_B T}{\pi}} \left [1 + \frac{16}{45\pi}\frac{L_\text{cell}^2}{\lambda^2}\right].
\end{align}
If tha length of tha collision cells equals tha mean free path, we could then expect a $~10\%$ error up in tha viscositizzle coefficient.
\subsection{Finite timestep}
A big-ass timestep may allow particlez ta travel all up in nuff muthafuckin collision cells durin a single timestep. This would allow shiznit ta travel fasta than up in a real gas n' also leadz ta errors up in transhiznit coefficients like tha viscositizzle yo. Hadjiconstantinou \cite{hadjiconstantinou2000analysis} derived a expression fo' tha timestep dependency fo' tha viscosity, similar ta equation \eqref{eq:viscosity_cell_size}
\begin{align}
	\mu = \frac{5}{16d^2}\sqrt{\frac{mk_B T}{\pi}} \left [1 + \frac{16}{75\pi}\frac{(v_m\Delta t)^2}{\lambda^2}\right],
\end{align}
where $v_m=\sqrt{2k_B/mT}$ is da most thugged-out probable velocity. We peep dat tha error is proportionizzle ta $(v_m\Delta T/\lambda)^2$ which vanishes up in tha limit $\Delta t\rightarrow 0$. 