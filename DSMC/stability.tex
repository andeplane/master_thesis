\section{Numerical stability and discretization error}
\label{sec:dsmc_stability}
Most numerical methods have a critical stability criterion where the energy or some other property might diverge if the timestep is too large. For example, while solving PDE's with a finite difference scheme, we often encounter the Courant number which is a critical threshold of the ratio of the discretization length of space and time. For the one-dimensional wave equation, this can be expressed as
\begin{align}
	C = \frac{|\dot x|_{max} \Delta x}{\Delta t} \leq C_{max},
\end{align}
where $|\dot x|_{max}$ is the magnitude of the velocity. If the spatial grid has high resolution, small $\Delta x$, we need a similarly small timestep $\Delta t$.\\
However, since the DSMC model always conserves energy and momentum (during particle collisions), the method is in principle numerically stable for any timestep. As mentioned in section \ref{sec:dsmc_model}, the timestep is splitted into two parts; moving and colliding. The timestep should therefore be smaller than the mean collision time. Larger timesteps may result in large errors in the transport coefficients (such as viscosity and thermal conductivity)\cite{karniadakis2005microflows}. While the timestep is compared to the mean collision time $\tau_\text{coll}$, the collision cell size can be seen as the spatial discretization, and be compared to the mean free path $\lambda$. 
\subsection{Finite cell size}
The collision cells allows all particles within a cell to collide with each other. So if the cell size is very large, particles from a hot region (in one corner of the collision cell) may collide with particles in a colder region (maybe in another corner) that are displaced by a large distance. This could enable heat to transfer much faster than it would in a real gas. The cell size $L_\text{cell}$ should therefore at least be smaller than the mean free path\cite{karniadakis2005microflows}. The viscosity can be calculated from kinetic theory
\begin{align}
	\mu = \frac{5}{16d^2}\sqrt{\frac{mk_B T}{\pi}},
\end{align}
which Garcia et al. \cite{alexander1998cell} used to show that the error in the viscosity has a quadratic dependency of the cell size
\begin{align}
	\label{eq:viscosity_cell_size}
	\mu(L_\text{cell}) = \frac{5}{16d^2}\sqrt{\frac{mk_B T}{\pi}} \left [1 + \frac{16}{45\pi}\frac{L_\text{cell}^2}{\lambda^2}\right].
\end{align}
If the length of the collision cells equals the mean free path, we could then expect a $~10\%$ error in the viscosity coefficient.
\subsection{Finite timestep}
A large timestep may allow particles to travel through several collision cells during a single timestep. This would allow information to travel faster than in a real gas and also leads to errors in transport coefficients like the viscosity. Hadjiconstantinou \cite{hadjiconstantinou2000analysis} derived an expression for the timestep dependency for the viscosity, similar to equation \eqref{eq:viscosity_cell_size}
\begin{align}
	\mu = \frac{5}{16d^2}\sqrt{\frac{mk_B T}{\pi}} \left [1 + \frac{16}{75\pi}\frac{(v_m\Delta t)^2}{\lambda^2}\right],
\end{align}
where $v_m=\sqrt{2k_B/mT}$ is the most probable velocity. We see that the error is proportional to $(v_m\Delta T/\lambda)^2$ which vanishes in the limit $\Delta t\rightarrow 0$. 