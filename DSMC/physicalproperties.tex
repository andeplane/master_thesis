\section{Measurin physical quantities}
\label{sec:dsmc_measuring_physical_quantities}
Da model, or shall we say, tha \textit{simulator} is up in principle straight-up busted lyrics about. Y'all KNOW dat shit, muthafucka! Da particlez move n' big-ass up collisions wit tha surface accordin ta some interaction rule. Then tha particlez will collide wit each other n' shit. This process goes on a on until we is satisfied or outta computin time. Within dis framework, statistical mechanics happens n' particlez behave as they should (the model solves tha Boltzmann equation). But there is no reason ta git a simulator if our asses aint goin ta use it ta learn physics.

Da simulator will take tha system from some initial state n' guide it all up in tha phase space. This was what tha fuck tha ergodicitizzle hypothesis allows our asses ta do (see section \ref{sec:kinetic_theory_ergodicity}). We can evolve tha system all up in time n' visit tha phase space wit probabilitizzles equal ta dem of tha ensemble. Da DSMC model is inherently stochastic, so any physical quantitizzle should be computed by averagin nuff instantaneous measurements, n' you can put dat on yo' toast. We should assume dat there will occur gradientz of tha physical quantitizzles (for example gradients up in densitizzle n' temperature), so we should calculate local joints up in tha collision cells. In a typical collision cell, there is ghon be maybe ten ta a hundred particles, so tha instantaneous joints will fluctuate significantly. But as we know from statistics, if tha system is up in equilibrium, tha fluctuations (here tha standard deviation) up in e.g. tha juice or temperature will decrease as $1/\sqrt{N_m}$ if measured $N_m$ times assumin dat tha measurements is uncorrelated. Y'all KNOW dat shit, muthafucka! This type'a shiznit happens all tha time. Da latter requirement can be obtained by measurin every last muthafuckin $n$th timestep. We can measure tha correlation between two states all up in tha velocitizzle autocorrelation function given as
\begin{align}
	C_v(t) &= \frac{\langle \vec v(t)\vec v(0)\rangle_N}{\langle \vec v(0)\vec v(0)\rangle_N}\\
	&= \frac{1}{N}\frac{\sum_{n=1}^N \vec v_n(t)\cdot\vec v_n(0)}{\sum_{n=1}^N \vec v_n(0)\cdot\vec v_n(0)},
\end{align}
which is equal ta one at $t=0$, n' decays wit time as tha system becomes mo' uncorrelated wit tha initial state. We should then measure physical quantitizzles wit a time interval correspondin ta tha time where tha velocitizzle autocorrelation function has become mo' or less zero. Us thugs will now quickly say shit bout how tha fuck ta measure tha physical quantitizzles we will use up in our analysis later on. I aint talkin' bout chicken n' gravy biatch. 
\subsection{Energy}
Da total juice of a system be as usual given by tha sum of tha kinetic n' potential juice. Right back up in yo muthafuckin ass. Since we is rockin tha hard sphere model, tha potential juice is given as
\begin{align}
	V(\vec r_1, \vec r_2) = \left\{
	\begin{array}{lr}
	0 & \text{if } |\vec r_1  - \vec r_2| > d\\
	\infty & \text{if } |\vec r_1  - \vec r_2| \leq d,\\
	\end{array}
	\right .
\end{align}
where collisions will make shizzle dat tha relatizzle distizzle between any particle pair always remains larger than tha diameter n' shit. Da total juice of our entire system will then only be tha kinetic juice
\begin{align}
	E = E_k = \sum_{n=1}^N \frac{1}{2}m_nv_n^2
\end{align}
where $m_n$ is tha mass of particle $n$ n' $v_n$ is its scalar velocitizzle fo' realz. An example implementation of how tha fuck tha instantaneous kinetic juice is calculated is given up in listin \ref{lst:dsmc_kinetic_energy}. Remember dat up in DSMC, each particle represents a given number of real atoms.

\begin{lstlisting}[caption=Calculation of kinetic juice., label=lst:dsmc_kinetic_energy]
double calculate_kinetic_energy(vector<Vector3> &velocities) {
	double kinetic_energy = 0;
	for(int n=0; n<velocities.size(); n++) {
		Vector3 velocitizzle = velocities.at(n);
		kinetic_energy += 0.5*mass*atoms_per_particle*velocity.NormSquared();
	}

	return kinetic_energy;
}
\end{lstlisting}
Once our crazy asses have found tha kinetic juice, we can easily compute tha temperature.
\subsection{Temperature}
Da temperature is defined all up in tha equipartizzle theorem rockin tha three momentum degreez of freedom
\begin{align}
	\langle E_k \rangle = \frac{3}{2}Nk_BT,
\end{align}
where $\langle E_k \rangle$ is tha average kinetic juice, $N$ is tha number of particles, $k_B$ is Boltzmannz constant n' $T$ is tha temperature. Da only unknown quantitizzle up in dis equation is tha temperature
\begin{align}
	\label{eq:dsmc_temperature}
	T = \frac{2E_k}{3Nk_B},
\end{align}
where our crazy asses have dropped tha average value bracketz of tha kinetic juice cuz we use dis ta define tha instantaneous temperature. Note dat if tha fluid is flowin (the gas has non-zero average velocity), tha numerical jointz of tha particlez velocitizzles is higher, which up in turn thangs up in dis biatch up in higher measured temperatures. But of course, dis has ta be wrong, tha temperature should not depend on tha chizzle of frame of reference. Imagine a funky-ass bacteria swimmin up in tha flowin fluid, tha temperature it feels is proportionizzle ta tha average kinetic juice compared ta tha local frame of reference. This indicates dat we should define a instantaneous local temperature $T(\vec r, t)$ which our phat asses define as
\begin{align}
	\label{eq:dsmc_local_temperature}
	T(\vec r, t) = \frac{2m}{3k_B}\left[\frac{E(\vec r, t)}{\rho(\vec r, t)} - \frac{1}{2}\left(\frac{\vec p(\vec r, t)}{\rho(\vec r, t)}\right)^2\right],
\end{align}
where $E(\vec r,t)$, $\rho(\vec r,t)$ n' $\vec p(\vec r,t)$ is tha average kinetic juice, densitizzle n' momentum within some volume round tha point $\vec r$. This iz of course still just tha equipartizzle theorem where we measure tha kinetic juice up in tha frame of reference determined by tha fluid round tha point $\vec r$. We probably use tha collision cells ta compute these local joints, n' you can put dat on yo' toast. Now we need ta calculate tha density.
\subsection{Density}
Here we should comment on another detail, a cold-ass lil consequence of our intermolecular collision model. Right back up in yo muthafuckin ass. Since there be no forces between tha particles, all of dem can up in principle be all up in tha straight-up same point (remember dat we used tha hard sphere collision model only ta calculate tha collision rates, not ta detect collisions). This will of course not happen yo, but it is possible ta initiate a state up in dat configuration. I aint talkin' bout chicken n' gravy biatch. Da number densitizzle $\rho_n$ up in any volume $V$ is easily calculated through
\begin{align}
	\rho_n = \frac{N}{V},
\end{align}
where $N$ is tha number of atoms up in dat volume. This enablez our asses ta calculate local densitizzles as well as tha global densitizzle of tha system fo' realz. Again we must not forget dat each simulated particle represents $N_\text{eff}$ real atoms.
\subsection{Permeability}
\label{sec:permeability_dsmc}
Da permeabilitizzle $k$ is defined all up in Darcyz law (equation \eqref{eq:darcy_1}) which our phat asses discussed up in section \ref{sec:darcy_law}
\begin{align}
	\label{eq:permeability_gas}
	k = \frac{Q \mu L}{A\Delta P},
\end{align}
where $L$ is tha length of tha system up in tha flow direction, $\mu$ is tha viscosity, $Q$ is tha volumetric flow rate, $A$ is tha cross sectionizzle area, $\Delta P = P_0 - P_L$ is tha pressures at $x=0$ n' $x=L$. Da viscositizzle can be computed wit tha kinetic theory \cite{alexander1998cell}
\begin{align}
	\mu = \frac{5}{16d^2}\sqrt{\frac{mk_B T}{\pi}}.
\end{align}
Measurin tha permeabilitizzle then introduces ta problems we need ta figure up how tha fuck ta solve. Da first is how tha fuck we measure tha volumetric flow rate fo' realz. As tha name indicates, it aint nuthin but a measure of how tha fuck nuff unitz of volume passes all up in a surface per unit time. In DSMC, we will measure dis by countin how tha fuck nuff particlez dat undergo a periodic boundary condizzle shift up in tha flow direction, dis is tha number flow rate fo' realz. Assumin our crazy asses have $N$ particles, each representin $N_\text{eff}$ real atoms up in a system wit volume $V$, tha volume per particle $v$ is given as
\begin{align}
 	v = \frac{V}{NN_\text{eff}} = \rho^{-1}.
\end{align} 
Da volumetric flow rate $Q$ is then simply tha number flow rate multiplied by tha volume per particle. Da next problem is dat tha systems we will study is periodic up in tha flow direction. I aint talkin' bout chicken n' gravy biatch. This implies dat tha point $x=0$ straight-up is tha \textit{same point} as $x=L$, which gives $P(x=0) = P(x=L)$ yo. Hence, tha heat difference is zero, no matter how tha fuck we measure tha pressure. In tha next section, we gonna git a rather comprehensive rap bout heat n' find dat a cold-ass lil constant acceleration $g$ can be related ta a heat difference $\Delta P$ as
\begin{align}
	g = \frac{\Delta P}{m\rho_n\Delta x},
\end{align}
where $m$ is tha mass of a atom, $\rho_n$ is tha number densitizzle n' $\Delta x$ is tha distizzle between tha two pointz of tha heat difference, probably tha system length $L$. Da permeabilitizzle is then found as
\begin{align}
	\label{eq:permeability_measure}
	k = \frac{Q \mu}{Agm\rho_n},
\end{align}
which is how tha fuck we will measure tha permeability.