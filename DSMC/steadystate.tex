\section{Reachin a steady state}
\label{sec:dsmc_steady_state}
Since we wanna study flow up in nanoporous media, before inducin tha flow, tha fluid is on average obviously at rest. Immediately afta our crazy asses have started applyin tha constant force dat will make tha fluid flow, tha fluid velocitizzle is still approximately zero fo' realz. After a cold-ass lil certain amount of time, tha system will reach a steady state which up in its most simple form can be defined as when tha time derivatizzle of tha \textit{fluid velocity} up in any region, tha local velocity, is zero. We should not start ta sample flow statistics like tha permeabilitizzle until such a state has been reached. Y'all KNOW dat shit, muthafucka! But fuck dat shiznit yo, tha word on tha street is dat tha system may not be up in a steady state even though tha average local fluid velocitizzle do not chizzle over time. There is other physical quantitizzles like dat may still be changing.

A naive yo, but simple approach ta measure whether or not tha fluid velocitizzle has converged is ta peep tha measured temperature defined up in equation \eqref{eq:dsmc_temperature}. If tha gas temperature starts up at $T= $\unit{300}{\kelvin} before tha flow is induced, tha measured temperature will increase while tha fluid velocitizzle increases. Once tha fluid has reached a steady state, tha temperature gonna git converged ta some value it will continue fluctuatin around. Y'all KNOW dat shit, muthafucka! For simplicity, dis is how tha fuck our crazy asses have determined whether or not tha system has reached tha steady state. In future pimpment of tha code, mo' betta methodz should be implemented.