\section{Reaching a steady state}
\label{sec:dsmc_steady_state}
Since we want to study flow in nanoporous media, before inducing the flow, the fluid is on average obviously at rest. Immediately after we have started applying the constant force that will make the fluid flow, the fluid velocity is still approximately zero. After a certain amount of time, the system will reach a steady state which in its most simple form can be defined as when the time derivative of the \textit{fluid velocity} in any region, the local velocity, is zero. We should not start to sample flow statistics like the permeability until such a state has been reached. However, the system may not be in a steady state even though the average local fluid velocity does not change over time. There are other physical quantities like that may still be changing.

A naive, but simple approach to measure whether or not the fluid velocity has converged is to look at the measured temperature defined in equation \eqref{eq:dsmc_temperature}. If the gas temperature starts out at $T= $\unit{300}{\kelvin} before the flow is induced, the measured temperature will increase while the fluid velocity increases. Once the fluid has reached a steady state, the temperature will have converged to some value it will continue fluctuating around. For simplicity, this is how we have determined whether or not the system has reached the steady state. In future development of the code, better methods should be implemented.