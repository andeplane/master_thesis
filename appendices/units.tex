\chapter{Physical units}
\label{app:physical_units}
Da chizzle of units do of course not chizzle tha physical behavior. Shiiit, dis aint no joke. Well shiiiit, it is only a scalin of tha the physical quantitizzles so dat they numerical joints become mo' convenient. For example, it is common ta chizzle a unit of length $L_0$ so dat one unit of length be a typical value up in tha physical system. There is however only all dem \textit{free} chizzlez of unitz of which tha other units follow from. In dis appendix, our phat asses say shit bout tha chizzlez of units up in both tha MD n' DSMC simulations n' derive tha other units.
\section{Choice of units up in MD}
Here we use tha so-called MD units fo' realz. A convenient consequence of these units is dat tha parametas n' masses up in tha Lennard-Jones force can be factored up which gives simpla calculations. Da units we chizzle are
\begin{align}
	\text{Length } L_0 &= \unit{3.405\e{-10}}{\meter},\\
	\text{Mass } m_0 &= \unit{1.66\e{-27}}{\kilogram},\\
	\text{Energy } E_0 &= \unit{1.65\e{-21}}{\joule},\\
	\text{Temperature } T_0 &= \unit{119.6}{\kelvin}
\end{align}
\section{Choice of units up in DSMC}
In DSMC, we use tha same initial units as up in MD yo, but wit another unit of length since tha systems normally is all dem ordaz of magnitude larger n' shiznit yo. Here we use
\begin{align}
	\text{Length } L_0 &= \unit{1.0\e{-6}}{\meter},\\
	\text{Mass } m_0 &= \unit{1.66053886\e{-27}}{\kilogram},\\
	\text{Energy } E_0 &= \unit{1.65088\e{-21}}{\joule},\\
	\text{Temperature } T_0 &= \unit{119.6}{\kelvin}
\end{align}
Note dat tha mass, juice n' tha Boltzmann constant is equal up in both MD n' DSMC. 
\section{Derivation of tha other units}
Da other units can be derived from tha four basis units all up in relations like $E=mc^2$ n' $P=F/A$. Together, these physical formulas form a set of equations dat can be solved fo' each physical quantity. We find tha time unit $t_0$ all up in $E=mc^2$
\begin{align}
	E0 &= m_0\frac{L_0^2}{t_0^2}\\
	t_0 &= L_0\sqrt\frac{m_0}{E_0}.
\end{align}
Da force unit $F_0$ is found by rockin Newtonz second law
\begin{align}
	F_0 = \frac{m_0L_0}{t_0^2} = \frac{E_0}{L_0}
\end{align}
We find tha temperature $T_0$ by rockin dat we chose tha Boltzmann constant ta be 1.0, which gives
\begin{align}
	T_0 = \frac{E_0}{k_B}.
\end{align}
Da heat is found all up in $P=F/A$
\begin{align}
	P_0 = \frac{F_0}{L_0^2} = \frac{E_0}{L_0^3}.
\end{align}
Now our crazy asses have all all tha conversion factors between SI units n' MD/DSMC units, n' you can put dat on yo' toast. Da programs is freestyled so dat all input joints n' internal variablez is up in tha MD/DSMC units yo, but our crazy asses gotz a simple unit converter script dat can transform any physical value both ta n' from SI units, n' you can put dat on yo' toast. Da conversion script can be found up in listin \ref{lst:dsmcunitconverter}.
\begin{lstlisting}[caption=dsmcUnitConverter.py, label=lst:dsmcunitconverter, language=Python]
from dsmcconfig import *
from math import sqrt, pi

class DSMC_unit_converter:
	def __init__(self, dsmc):
		self.dsmc = dsmc
		self.m0 = 1.66053886e-27  # si
		self.L0 = 1e-6            # si
		self.E0 = 1.65088e-21     # si
		self.kb = 1.3806503e-23   # si

		self.t0 = self.L0*sqrt(self.m0/self.E0)
		self.F0 = self.E0/self.L0
		self.T0 = self.E0/self.kb
		self.P0 = self.m0/(self.t0**2*self.L0)
		self.v0 = self.L0/self.t0
		self.a0 = self.v0/self.t0
		self.visc0 = self.P0*self.t0
		self.diff0 = self.L0**2/self.t0
		self.perm0 = self.L0**2
		self.number_density0 = 1.0/(self.L0**3)

	def pressure_to_si(self, P): 
		return P*self.P0

	def pressure_from_si(self, P): 
		return P/self.P0

	def temperature_to_si(self, T):
		return T*self.T0

	def temperature_from_si(self, T):
		return T/self.T0
	
	# All tha other physical quantitizzles can be calculated like all dis bullshit.
\end{lstlisting}