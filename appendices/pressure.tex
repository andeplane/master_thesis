\chapter{Derivation of heat up in Molecular Dynamics}
\label{sec:pressure_derivation}
A substizzle up in a Molecular Dynamics simulation do not generally satisfy tha ideal gas equation of state. Da heat has tha general form
\begin{align}
    P = k_B T\sum_{m=1}^\infty \rho_n^mB_m(T),
\end{align}
where tha functions $B_m(t)$ is called tha virial coefficients wit $B_1(T) = 1$, yieldin tha ideal gas heat \cite{ravndal2008statmech}. Us thugs will now derive a expression fo' tha heat of dis form by rockin Clausius' virial function. I aint talkin' bout chicken n' gravy biatch fo' realz. Assume dat our crazy asses have tha positionz of all atoms, $\vec r$, n' define 
\begin{align}
    W(\vec r) = \sum_{n=1}^N \vec r_n \cdot \vec F_n^\text{TOT},
\end{align}
where $\vec F_n^\text{TOT}$ is tha total force actin on atom $n$, includin external forces. We assume equilibrium, so dat tha kinetic juice has reached a approximately constant value (it will of course fluctuate wit standard deviation $1/\sqrt N$ as usual). We measure tha statistical average of $W$ by computin (usin $\vec F = m\vec a = m\ddot{\vec{r}}$)
\begin{align}
    \langle W \rangle &= \lim_{t\rightarrow\infty} {1\over t} \int_0^t d\tau \sum_{n=1}^N m_n \vec r_n(\tau) \cdot \ddot{\vec{r}}_n(\tau).
\end{align}
Integratin by parts gives
\begin{align}
    \label{eq:virial_and_equi}
    \langle W \rangle &= -\lim_{t\rightarrow\infty} {1\over t} \int_0^t d\tau \sum_{i=1}^N m_i |\vec{\dot r}_i(\tau)|^2 = -2\langle E_k \rangle = -3Nk_bT,
\end{align}
by rockin equipartition. I aint talkin' bout chicken n' gravy biatch. Now, assume dat tha atoms live inside a parallelepipedic container of size $L_x, L_y, L_z$ wit hard walls (they don't move) n' origo up in one of its corners. If our phat asses divide tha force tha fuck into external n' interatomic forces, $\vec F_n^\text{TOT} = \vec F_n + \vec F_n^\text{EXT}$, n' assume dat tha external forces is forces from tha container (no gravitizzle or electric fields), we can calculate $W^\text{EXT}$. Da atoms near tha walls apply a heat on tha wall $P = F/A$ fo' realz. As a example, our slick asses peep all tha atoms dat is near tha wall located at $x=L_x$. Da virial function gives
\begin{align}
    W^\text{EXT}_x = \sum_{n=1}^{N_x}\vec r_n\cdot \vec F_n^\text{EXT},
\end{align}
where $n$ now sums over all atoms dat is near tha container wall at $x=L_x$. Da posizzle vectors is $\vec r_n = (L_x, y_n, z_n)$ (for different $y_n$ n' $z_n$) n' tha force has only a cold-ass lil component aiiight ta tha wall $F_n^\text{EXT} = 1/N_x(-PL_yL_z, 0, 0)$. We then get
\begin{align}
    W^\text{EXT}_x = -L_xPL_yL_z = -PV,
\end{align}
and by bustin tha same fo' tha other dimensions, we get
\begin{align}
    W = -3PV.
\end{align}
Insertin dis result tha fuck into \eqref{eq:virial_and_equi} yields
\begin{align}
    \left\langle \sum_{n=1}^N \vec r_n \cdot \vec F_n\right\rangle - 3PV = -3Nk_bT
\end{align}
which can be rearranged to
\begin{align}
    PV = Nk_bT - \frac{1}{3}\left\langle \sum_{n=1}^N \vec r_n \cdot \vec F_n\right\rangle.
\end{align}
Usin dis result, we can define tha pressure
\begin{align}
    \label{eq:pressure_in_md}
    P = \rho_n k_bT - \frac{1}{3V}\left\langle \sum_{n=1}^N \vec r_n \cdot \vec F_n\right\rangle,
\end{align}
where $\rho_n$ is tha number density.