\chapter{Da Liouville operator n' time integrators}
\label{app:liouville}
Da derivation of time integrator schemes is probably done up in a mathematical sense rockin Tay-Tay expansions. Given a Tay-Tay expansion $f(x+h) = f(x) + hf'(x) + h^2/2f''(x) + ...$, we often truncate at some term, yieldin a truncation error $O(h^n)$. In physics, there be other propertizzlez of a system of which tha error don't scale as tha truncation error fo' realz. A typical example is tha juice of a particle system. Two different time integrators may have tha same truncation error yo, but have straight-up different long term behavior if one has a thugged-out drift up in tha juice whereas tha other don't.

There is other propertizzles dat might be hard as fuck ta measure all up in a mathematical analysiz of tha Tay-Tay expansion dat may be mo' blingin than tha truncation error itself. In tha Hamiltonian formulation of old-ass mechanics, we obtain tha equationz of motion all up in tha juice operator $\vec H = \vec T + \vec V$, where $\vec T$ n' $\vec V$ is operators measurin tha kinetic n' potential juice. Usin tha physical description of a system can help pimpin mo' betta integration schemes.

In dis chapter we will address tha equationz of motion by rockin tha Liouville operator ta derive a way ta create time integration schemes. We start by lookin all up in tha phase space coordinates n' define tha Liouville operator up in section \ref{sec:liouville_operator}. We then define tha time evolution operator n' split tha Liouville operator tha fuck into two operators; one operator actin on tha positions n' one actin on tha momenta.  These operators do not commute, so we use tha Trotter identitizzle ta introduce time discretization n' derive tha Velocitizzle Verlet algorithm, which is tha one used up in tha Molecular Dynamics code. We then do a analysiz of tha mathematical error ta find tha local error (which is tha same ol' dirty as tha truncation error) up in section \ref{sec:velocity_verlet_error}. This derivation is done as up in \cite{frenkel2001understanding}. 

%In section \ref{sec:multiple_timestep_schemes} we use tha Liouville formulation ta derive a multiple timestep integration scheme dat can be used if tha Molecular Dynamics system consistz of both light n' heavy atoms. Light atoms like hydrogen need a lil' small-ass timestep up in order ta accurately integrate vibrationz of high frequency, whereas tha heavier atoms tolerate a higher timestep. 
\section{Liouville operator}
\label{sec:liouville_operator}
Da physical system consistz of $N$ particles, each havin three positions n' three momenta definin tha phase space point $(\vec r, \vec p)$. Now assume some function of these variablez $f(\vec r(t), \vec p(t)) = f(t)$ (the function is indirectly a gangbangin' function of time) dat has tha time derivative
\begin{align}
	\dot f(t) = \dot {\vec r}\frac{\partial f(t)}{\partial \vec r} + \dot {\vec p}\frac{\partial f(t)}{\partial \vec p} \equiv \liouville f(t),
\end{align}
where our crazy asses have defined tha Liouville operator
\begin{align}
	\liouville = \dot{\vec r}\frac{\partial }{\partial \vec r} + \dot{\vec p}\frac{\partial }{\partial \vec p}.
\end{align}
This allows our asses ta define tha time evolution operator $\hat{\mathcal U}(t)$
\begin{align}
	\label{eq:liouville_time_evolution}
	f(t) = \hat{\mathcal U}(t)f(0) = e^{\liouville t}f(0),
\end{align}
which is easily verified
\begin{align}
	\dot f(t) = \liouville \left[e^{\liouville t}f(0)\right] = \liouville f(t).
\end{align}
If we now split tha Liouville operator tha fuck into two parts
\begin{align}
	\liouville = \liouviller + \liouvillep,
\end{align}
so that
\begin{align}
	\liouviller = \dot{\vec r(0)}\frac{\partial }{\partial \vec r}.
\end{align}
Let our asses peep what tha fuck dis operator can do if we bang it tha fuck into equation \eqref{eq:liouville_time_evolution} n' expand tha exponential
\begin{align}
	f(t) &= \exp\left(\liouviller t\right) f(0)\\
	&= \exp\left(\dot{\vec r}t\frac{\partial }{\partial \vec r}\right)f(0)\\
	&= \sum_{n=0}^\infty \frac{(\dot{\vec r}(0)t)^n}{n!}\frac{\partial^n}{\partial \vec r^n} f(0)\\
	&= f\left[\left(\vec r(0) + \dot{\vec r}(0)t\right), \vec p(0)\right].
\end{align}
It do just what tha fuck we expect it ta do, it aint nuthin but a gangbangin' finger-lickin' displacement operator, movin tha points up in tha phase space accordin ta they time derivative. Da momentum Liouville operator do of course exactly tha same, so dat by applyin tha total time evolution operator, our phat asses do indeed get
\begin{align}
	f(t) &= e^{\liouville t}f(0) = f\left[\left(\vec r(0) + \dot{\vec r}(0)t\right), \left(\vec p(0) + \dot{\vec p}(0)t\right)\right].
\end{align}

If we was ta use dis up in a simulation, we normally do not apply tha full operator all up in tha same time, we might first treat tha positions, then tha momenta. Right back up in yo muthafuckin ass. So ideally, we would wanna first apply one operator, then tha next one
\begin{align}
	e^{\liouville} = e^{\liouvillep + \liouviller} \neq e^{\liouvillep}e^{\liouviller},
\end{align}
but dis aint tha case since tha operators do necessarily commute. But fuck dat shiznit yo, tha word on tha street is dat fo' two operators $\vec A$ n' $\vec B$, we can use tha \textit{Trotter identity}\cite{frenkel2001understanding}
\begin{align}
	e^{A + B} = \lim_{N\rightarrow\infty}\left(e^{A/2M}e^{B/M}e^{A/2M}\right)^N,
\end{align}
which can be truncated
\begin{align}
	e^{A + B} = \left(e^{A/2N}e^{B/N}e^{A/2N}\right)^Ne^{O(1/N^2)}.
\end{align}
This can be used ta derive different time integrator schemes. Us thugs will now derive tha Velocitizzle Verlet scheme which is used up in tha Molecular Dynamics code.
\section{Derivation of tha Velocitizzle Verlet algorithm}
Da truncated Trotter identitizzle is neat, we can replace $A$ wit tha $\liouvillep$ n' $B$ wit $\liouviller$
\begin{align}
	\frac{\liouvillep}{M} &\equiv \Delta t\dot{\vec p}(0)\frac{\partial}{\partial \vec p}\\
	\frac{\liouviller}{M} &\equiv \Delta t\dot{\vec r}(0)\frac{\partial}{\partial \vec r}
\end{align}
definin tha timestep $\Delta t = t/N$. Da truncated time evolution operator now reads
\begin{align}
	\hat {\mathcal U} (t) = \left(e^{\liouvillep \Delta t/2}e^{\liouviller \Delta t}e^{\liouvillep \Delta t/2}\right)^N
\end{align}
where we identify one timestep iteration
\begin{align}
	\hat {\mathcal U} (\Delta t) = e^{\liouvillep \Delta t/2}e^{\liouviller \Delta t}e^{\liouvillep \Delta t/2}.
\end{align}
If we now apply dis operator on $f(0)$, we first apply tha rightmost operator
\begin{align}
	e^{\liouvillep \Delta t/2}f\left[\vec p(0), \vec r(0)\right] = f\left\{\vec r(0), \left[\vec p(0) + \frac{\Delta t}{2}\dot{\vec p}(0)\right] \right\},
\end{align}
before applyin $\exp(\liouviller t)$
\begin{align}
	e^{\liouviller t}f\left\{\vec r(0), \left[\vec p(0) + \frac{\Delta t}{2}\dot{\vec p}(0)\right] \right\}\\
	= f\left\{\left[\vec r(0) + \Delta t \dot{\vec r}(\Delta t/2)\right], \left[\vec p(0) + \frac{\Delta t}{2}\dot{\vec p}(0)\right] \right\}.
\end{align}
Our last operator gives tha final result
\begin{align}
	f\left\{\left[\vec r(0) + \Delta t \dot{\vec r}(\Delta t/2)\right], \left[\vec p(0) + \frac{\Delta t}{2}\dot{\vec p}(0) + \frac{\Delta t}{2}\dot{\vec p}(\Delta t)\right] \right\}.
\end{align}
These steps can be summarized as one probably do wit time integrators
\begin{align}
	\vec v(\Delta t/2) &= \vec v(0) + \frac{\vec F(0)}{m}\frac{\Delta t}{2}\\
	\vec r(\Delta t) &= \vec r(0) + \vec v(\Delta t/2)\Delta t\\
	\vec v(\Delta t) &= \vec v(\Delta t/2) + \frac{\vec F(\Delta t)}{m}\frac{\Delta t}{2},
\end{align}
where our crazy asses have replaced $\dot{\vec p}$ wit tha equivalent $(\vec F/m)$ n' $\dot{\vec r}$ wit $\vec v$ which is valid if tha forces can be calculated from tha position. I aint talkin' bout chicken n' gravy biatch. These steps is called tha Velocitizzle Verlet algorithm n' has, as we now will see, a error $O(\Delta t^3)$.

\section{Truncation error}
\label{sec:velocity_verlet_error}
Durin one timestep iteration, we approximate tha Liouville operator
\begin{align}
	e^{\liouville \Delta t} \approx e^{\liouvillep \Delta t/2}e^{\liouviller \Delta t}e^{\liouvillep \Delta t/2} = e^{\liouville \Delta t + \hat{\vec \epsilon}},
\end{align}
where our crazy asses have introduced tha \textit{error operator} $\hat{\vec \epsilon}$ dat represents our truncation error. Shiiit, dis aint no joke. These is linear operators on which we can use tha Campbell-Baker-Hausdorff expansion of general, non-commutin linear operators
\begin{align}
	e^{\lambda\hat{\vec A}}e^{\lambda\hat{\vec B}} = e^{\lambda\hat{\vec A} + \lambda\hat{\vec B} + \frac{\lambda^2}{2}[\hat{\vec A},\hat{\vec B}] + \frac{\lambda^3}{12}[\hat{\vec A},[\hat{\vec A},\hat{\vec B}]] + \frac{\lambda^3}{12}[\hat{\vec B},[\hat{\vec A},\hat{\vec B}]] + ...}
\end{align}
together wit 
\begin{align}
	e^{\oper A}e^{\oper B} = e^{\oper B + [\oper A,\oper B] + \frac{1}{2!}[\oper A,[\oper A,\oper B]] + \frac{1}{3!}[\oper A,[\oper A,[\oper A,\oper B]]] + ...}e^{\oper A}
\end{align}
to find tha leadin term up in $\hat{\vec \epsilon}$
\begin{align}
	-(\Delta t)^3\left(\frac{1}{24}[\liouviller, [\liouviller, \liouvillep]] + \frac{1}{12}[\liouvillep, [\liouviller, \liouvillep]]\right),
\end{align}
where we peep dat $\Delta t^3$ is tha global truncation error up in tha Velocitizzle Verlet algorithm afta $n$ timesteps. Da local truncation error is then $\Delta t^4$.
% \section{Multiple timestep integration schemes}
% \label{sec:multiple_timestep_schemes}
% Us thugs will now address a thang dat may occur up in a advanced Molecular Dynamics simulation. I aint talkin' bout chicken n' gravy biatch. Letz say we will simulate a system wit nuff muthafuckin atom types, like silica n' gin n juice n' shit. In such a system, our crazy asses have both hydrogen, oxygen n' silicon, which all may feel tha appearizzle of all tha other atom types yo. Hydrogen atoms is 28 times lighter than silicon. I aint talkin' bout chicken n' gravy biatch fo' realz. A consequence of dis is dat hydrogen gonna git much higher velocitizzles than tha other atom types, they will typically vibrate straight-up quickly up in tha gin n juice molecules. This will require a much smalla timestep than dem of tha silicon n' oxygen atoms would need. Y'all KNOW dat shit, muthafucka! Long range forces like tha Coulomb force is other examplez of interactions dat is accurate wit higher timesteps. Us thugs will now use tha Liouville formulation ta derive a scheme dat will allow multiple timesteps.\\
% Again we split tha Liouville operator tha fuck into tha sum of operators. \todo{Derive all dis bullshit.}